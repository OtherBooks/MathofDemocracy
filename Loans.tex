\section{Loans} \index{present value} \label{sec:Loans}

%\ifsolns Student Page 117 \fi
\begin{enumerate}
  %\item Your computer dies, so you go to Best Buy to get a new laptop.
        %Since you don't have enough money right now, you buy the laptop for \$820.00 on a credit card.
        %The credit card charges 12\% interest, compounded monthly.
        %What will happen if you pay \$141.49 each month?
        %
        %\begin{center}
					%\begin{tabular}{l}
          %Interest per month:\\
          %Monthly payment:
        %\end{tabular}
        %
        %\begin{tabular}{l*{3}{|p{0.74in}}|} \hline
				%Month & Loan Amount & Payment & Interest\\
				%\ifsolns
%0 &  \$820.00  & \$141.49  &  \$8.20 \\
%1 &  \$686.71  & \$141.49  &  \$6.87 \\
%2 &  \$552.09  & \$141.49  &  \$5.52 \\
%3 &  \$416.12  & \$141.49  &  \$4.16 \\
%4 &  \$278.79  & \$141.49  &  \$2.79 \\
%5 &  \$140.09  & \$141.49  &  \$1.40 \\
%6 &  \$ 0.00 &  &  \\\hline
%\else
          %\rule{0pt}{0.75cm} 0&&& \\ \hline
          %\rule{0pt}{0.75cm} 1&&& \\ \hline
          %\rule{0pt}{0.75cm} 2&&& \\ \hline
          %\rule{0pt}{0.75cm} 3&&& \\ \hline
          %\rule{0pt}{0.75cm} 4&&& \\ \hline
          %\rule{0pt}{0.75cm} 5&&& \\ \hline
          %\rule{0pt}{0.75cm} 6&&& \\ \hline\fi
        %\end{tabular}
%
				%\end{center}  
				%\item Notes on the time value of money:
        %\vfill
%\clearpage
  \item \boxedblank[2in]{\textbf{Present Value of an Annuity:} \color{magenta}\[ P = R\frac{1-(1+\frac rn)^{-nt}}{\frac rn}\] or
	\[ P = R\frac{1-(1+i)^{-m}}{i}\] 
	}
  \item The Present Value formula states how much a series of payments is worth \emph{as one lump sum at the beginning}.
        This can show up in several different ways; here are four.
        \begin{enumerate}
          \item \textbf{How much money do you need now to fund a series of future payments?} \\
                Example: You are planning to retire at age 68.  You will need a \$50,000 payment each year to live on,
                and you plan to be retired for 20 years.  The interest rate is 6\%.
                How much money do you need in your retirement account when you retire? \solution{\$573,496.06}
                \vfill
        \clearpage
          \item \textbf{What is a fair price to charge now in exchange for paying someone regularly in the future?} \\
				Example: You run a retirement home, and an elderly man named George wants to pay you a fixed sum now 	to have you take care of him for the rest of his life.  It will cost you \$2000 per month to take care of George,		and he will probably live another 15 years.  You can get interest rates of 5\% compounded monthly.		How much should you charge George? \solution{\$252,910.49}
				\vfill
          \item \textbf{What is a fair price to pay for something that will pay you regularly in the future?} \\
                Example: You are an investor looking at buying a copper mine.  
                The mine will produce an annual profit of \$1,200,000 per year for its owner for the next 20 years, but then the ore will run out.  Interest rates are sitting at 4\%.
                How much would be a fair price to pay for the mine? \solution{\$16,308,391.61}
                \vfill
          \clearpage
          \item \textbf{How does a loan amount relate to the regular loan payment?} \\
                Example: You put a \$2000 down payment on a \$12,000 car, financing the rest with a 5-year loan at 6\% interest, compounded monthly.
                \begin{enumerate}
                  \item How much will your monthly payments be? \solution{\$193.33}

                        \vfill
                  \item How much cash did you pay, total, for your \$12,000 car? \solution{\$13,599.68}
                        \vfill
                  \item How many dollars of interest did you pay for this car? \solution{\$1,599.68}
                        \vfill
                \end{enumerate}
                
        \end{enumerate}
\clearpage
  \item You have a student loan of \$9,000 to be paid back over 12 years at 4\% interest compounded monthly.
        \begin{enumerate}\setlength{\itemsep}{2in}
          \item How much will the monthly payment be? \solution{\$78.80}
          \item What will be the total amount you pay for your \$9,000 loan?  \solution{\$11,346.85}
          \item How much interest did you pay for this loan? \solution{\$2,346.85}

          \item If you want to pay it off in just 5 years, how much should you pay each month? \solution{\$165.75}
                How much interest will that save you? \solution{\$1,401.93}
        \end{enumerate}
  \clearpage
  \item You want to buy a car.  You have \$1,500 saved up for a down payment,
        and you can get a 5-year car loan at 3\% interest, compounded monthly.
        \begin{enumerate}
          \item If you can afford a \$200 monthly car payment, what is the most expensive car you can buy? \solution{\$12,630.47}
                \vfill
          \item If you want to buy an \$18,000 vehicle, what will the monthly payment be? \solution{\$296.48}
                \vfill
          \item If you buy that \$18,000 vehicle, how much will you pay in interest over the life of the loan? \solution{\$1,289.00}
                \vfill
        \end{enumerate}
\end{enumerate}

%</WORKSHEETS>
%%%%%%%%%%%%%%%%%%%%%%%%%%%%%%%%%%%%%%%%%%%%%%%%%%%%%%%%%%%%%%%%%%%%%%%%%%%%%%%%%%%%%%%%%%%%%%%%%%%
%<*HWHEADER>
\HOMEWORK
%</HWHEADER>

%<*HOMEWORK>

\begin{Fenumerate}

\item Find the present value of each of the following ordinary annuities.
      \begin{enumerate}
        \item Payments of \$500 made at the end of each quarter for  8 years, where 10\% annual interest is compounded quarterly.
              \solution*{$\$500 \dfrac{1-(1+\frac{.10}{4})^{-4\cdot 8}}{\frac{.10}{4}} = \boxed{\$10924.59.}$}\vfill
        \item Payments of \$100 made at the end of each month   for 10 years, where  6\% annual interest is compounded monthly.
              \ifsolns
                \par \soln
                  $\$100 \dfrac{1-(1+\frac{.06}{12})^{-12\cdot 10}}{\frac{.06}{12}} = \boxed{\$9007.35.}$
              \fi
              \studentsoln{\$9007.35}\vfill
      \end{enumerate}
      
\item Suppose you borrow \$16,000 from a bank to purchase a car.
      The bank charges 4\% annual interest, compounded monthly.
      You are to make equal monthly payments at the end of each of the next 48 months to amortize your loan.
      How much are your monthly payments?
      \ifsolns
        \par\soln
        We are looking for a payment $P$ such that
       \\ \centering{$ \displaystyle \$16{,}000 = P \dfrac{1-(1+\frac{.04}{12})^{-48}}{\frac{.04}{12}}.$}
        Solving we obtain $P=\boxed{\$361.26.}$
      \fi
      \studentsoln{\$361.26}\vfill

  \item You want to retire at age 65 with an \$80,000 annual income.
        You come from a long-lived family, so you want to be prepared in case you live to age 97.
        Interest rates are at 8\%.
        \begin{enumerate}
          \item How much money will you need in your retirement accounts when you retire, 
                in order to fund annual payments of \$80,000?
                \solution*{%
                  You need $97-65=32$ years of retirement income, so you need
                  $80000 \dfrac{1-(1+\frac{0.08}{1})^{-1\cdot 32}}{\frac{0.08}{1}} = \boxed{\$914{,}799.95.}$
                }\vfill
                %\studentsoln{\$914,799.95}
          \item If you start working at age 22, how much money should you deposit into your retirement account each month
                in order to have saved up the amount from part (a) by the time you retire at age 65?
                \solution*{%
                  You have $65-22 = 43$ working years, so you need $P$ such that$ \displaystyle \$914{,}799.95 = P \dfrac{(1+\frac{0.08}{12})^{12\cdot 43}-1}{\frac{0.08}{12}}.$
                  (This is the \emph{future value} formula, because you are \emph{saving} money.)
                  Solving, you find $P=\boxed{\$204.43.}$
                }\vfill
        \end{enumerate}


\hwnewpage
  \item You have to take out \$20,000 in student loans to get through college.  
        The interest rate is 6.8\% annually, compounded monthly, and you will make monthly payments for the next ten years.
        (No interest is charged or payments required until you leave school.)
        \begin{enumerate}
          \item How much will your monthly payment be?
                \solution*{\$230.16}\vfill
          \item How much interest will you pay over those ten years?  (Your answer should be in dollars.)
                \solution*{\$7619.20}\vfill
          \item Suppose you try to pay off the loan in just 5 years.
                How much will you have to pay each month to do so?
                \solution*{\$394.14}\vfill
          \item If you do pay off the loan in 5 years, how much interest will you pay total?
                \solution*{\$3648.40}\vfill
          \item How much money do you save by paying off the loan in 5 years instead of 10 years?
                \solution*{\$3970.80}\vfill
        \end{enumerate}

\hwnewpage
  
  \item Some time ago Steve Smith took out a mortgage from First National State Local Bank;
        his payment was \$900 per month. 
        FNSLB is strapped for cash these days, so it's considering ``selling the mortgage'' to Seventh National Bank; in other words, SNB will pay FNSLB a certain sum of money, and in return SNB will get
        all Steve Smith's remaining mortgage payments. \par
        There are 187 payments remaining on the mortgage, 
        and interest rates are at 4\% per year.
        How much should FNSLB charge Seventh National Bank to buy the mortgage?
        \solution*{\$125,088.21} \vfill
%\hwnewpage

\item You win a ``one million dollar'' lottery prize.  Hooray!
        \begin{enumerate}
          \item Suppose the lottery rules stipulate that you will be paid
                \$50{,}000 at the end of each of the next 20 years, 
                for a total of \$1{,}000{,}000 paid out.
                Assuming that annual interest rates will stay at 5\%, what is the present value of this prize?
                In other words, how much is this prize really worth \emph{today}?
                \solution*{%
                  You get an annuity of \$50,000 per year for 20 years.
                  The annuity is worth
                 $ \displaystyle \$50{,}000 \dfrac{1-(1+\frac{.05}{1})^{-20\cdot 1}}{\frac{.05}{1}} = \boxed{\$623{,}110.52.}$
                }\vfill
          \item Suppose instead the lottery rules stipulate that you will be paid
                \$50{,}000 today, and then \$50{,}000 at the end of each year for the next 19 years.
                Assuming 5\% interest rates,
                how much is \emph{this} prize really worth \emph{today}?
                \solution{
                  You get \$50,000 today, plus an annuity of \$50,000 per year for 19 years.
                  The annuity is worth
                 $ \displaystyle \$50{,}000 \dfrac{1-(1+\frac{.05}{1})^{-19\cdot 1}}{\frac{.05}{1}} = \$604{,}266.04,$
                  so adding in the \$50,000 you get paid now, the answer is
                 $ \displaystyle \$50{,}000 + \$604{,}266.04 = \boxed{\$654{,}266.04.}$
                }\vfill
        \end{enumerate}

\hwnewpage
\item Two oil wells are for sale.  
      The well in Varmint, TX promises to yield payments of \$6{,}000 at the end of each year for the next 10 years.
      The well in Mule's Ear, TX will     yield payments of \$4{,}000 at the end of each year for the next 20 years.
      (Notice that the lengths of time are different!)
      \begin{enumerate}
        \item Assuming that annual interest rates will hold steady at 8\% for the next 20 years, find the present value of each oil well.
              Which oil well is more valuable?\label{partFa}
              \ifsolns
                \par \soln
                The present value in Varmint is
                    $\$6000 \dfrac{1-(1+\frac{.08}{1})^{-1\cdot 10}}{\frac{.08}{1}} = \boxed{\$40260.49.}$
                The present value in Mule's Ear is
                    $\$4000 \dfrac{1-(1+\frac{.08}{1})^{-1\cdot 20}}{\frac{.08}{1}} = \boxed{\$39272.59.}$ \par
                Thus \fbox{Varmint has the higher present value.}
              \fi\vfill
        \item Assuming that annual interest rates will stay at 6\% for the next 20 years, find the present value of each oil well.
              Which oil well is more valuable?\label{partFb} \vfill
              \ifsolns
                \par \soln
                The present value in Varmint is
                    $\$6000 \dfrac{1-(1+\frac{.06}{1})^{-1\cdot 10}}{\frac{.06}{1}} = \boxed{\$44160.52.}$
                The present value in Mule's Ear is
                    $\$4000 \dfrac{1-(1+\frac{.06}{1})^{-1\cdot 20}}{\frac{.06}{1}} = \boxed{\$45879.68.}$
             \par   Thus \fbox{Mule's Ear has the higher present value.}
              \fi
        \item Why did the present values of the oil wells go up when the interest rates went down from part F\ref{partFa} to  part F\ref{partFb}?
              \ifsolns
                \par \soln
                If interest rates are lower, you would need to invest more money today to earn enough interest
                to make payments equivalent to what the oil wells will produce. \par
                \emph{(Note to grader: please don't grade this part of the problem.)}
              \fi\vfill
      \end{enumerate}



\end{Fenumerate} \ENDHOMEWORK
%%%%%%%%%%%%%%%%%%%%%%%%%%%%%%%%%%%%%%%%%%%%%%%%%%%%%%%%%%%%%%%%%%%%%%%%%%%%%%%%%%%%%%%%%%%%%%%%%%
\clearpage