\section{Affordability} \label{sec:Affordability}


\begin{enumerate}
  \item The amount banks generally require for a ``normal'' down payment is\ldots\vspace{0.5in} \ifsolns 10\%-20\% \else \fillwithlines{\stretch{1}}\fi
  \item \defnstyle{Private mortgage insurance} (PMI) is \ldots \fillwithlines{\stretch{1}} \ifsolns insurance payable to a lender or trustee for a pool of securities that may be required when taking out a mortgage loan.\fi
  \item PMI is always calculated as a percentage of the\ldots \index{Private Mortgage Insurance (PMI)}
        \ifsolns loan amount, not the purchase price\fi \vspace{0.5in}
  \item The credit union will let you get by with a 5\% down payment
        if you pay 0.6\% PMI per year.
        You still have \$11,000 in savings.
        \begin{enumerate}
          \item What is the most expensive house you can buy if you only pay a 5\% down payment? \solution{\$220,000.00}
                \vfill
          \item In this case, how much will you pay for PMI during the first year? \solution{\$1,254.00}
                \vfill
          \item How much will you pay each month for PMI? \solution{\$104.50}
                \vfill
        \end{enumerate}
\clearpage
	\item A \defnstyle{point} is \ldots  \index{mortgage!points} \ifsolns 1\% of the loan.\else \fillwithlines{\stretch{1}} \fi 
  \item \defnstyle{Closing costs} are what you have to pay at the time you take out the loan. \index{closing costs}
        There are several types, including the following:
        \begin{enumerate}
          \item A \defnstyle{down payment}.\fillwithlines{\stretch{1}} \index{mortgage!down payment}
          \item A \defnstyle{discount charge} is \fillwithlines{\stretch{1}} \index{mortgage!discount charge}
          \item A \defnstyle{loan origination fee} is \ldots  \index{mortgage!loan origination fee} \ifsolns An origination fee usually varies from 0.5\% (half a point) to 2\% (two points) of a given loan amount, depending on whether the loan was originated in the prime or the subprime market. For example, an origination fee of 2\% on a \$200,000 loan is  \$4,000.[2]\else \fillwithlines{\stretch{1}} \fi 
        \end{enumerate}
  
  \item \boxedblank{\textbf{Closing costs:}}
    \item You are buying a house for \$165,000 with a \$20,000 down payment,
          and the bank charges you \$253 in closing fees plus one point.
          How much are your closing costs? \solution{\$21,703.00}
            \vfill
%\clearpage
%    \item You want to buy a house.  You have a \$11,000 down payment saved up,
%          and you can get a 30-year mortgage at 5\% interest, compounded monthly.
%          \begin{enumerate}
%            \item If you can afford a \$600 monthly mortgage payment, what is the most expensive house you can buy?
%                  \vfill
%            \item If you want to buy a \$150,000 house, what will the monthly payment be?
%                  \vfill
%          \end{enumerate}
%\cleartoevenpage
  \item Your monthly housing expenses are often referred to as ``PITI.'' \fillwithlines{\stretch{1}} \par
        \mbox{\Huge P} \index{housing expenses!Principle} \par \solution{Principle} 
        \fillwithlines{\stretch{1}}
        \mbox{\Huge I} \index{housing expenses!Interest}\par \solution{Interest}
        \fillwithlines{\stretch{1}}
        {\Huge T} \index{housing expenses!Taxes}\par \solution{Taxes}
        \fillwithlines{\stretch{1}}
        {\Huge I} \index{housing expenses!Insurance} \solution{Insurance}
				\fillwithlines{\stretch{1}}
				\clearpage
				
  \item
  	Josef takes out a 30 year mortgage loan of \$50,000 with an APR of 6\%.
  	\begin{enumerate}
  	\item What is the monthly payment? \solution{\$299.78} \vfill
			\item How much does Josef owe on the mortgage after one month? \solution{\$49,950.22}\vfill
			\item How much of the first monthly payment went to paying off the loan (principle)?\solution{\$49.78}\vfill
			\item How much of the first monthly payment went to interest? \solution{\$250}\vfill
		\end{enumerate}
\pagebreak
  \item Juan and Regina Hernandez are buying a house selling for \$158,000.
        They will put 12\% down and get a 15-year mortgage at 3\% interest compounded monthly,
        but they will be charged 0.5\% PMI per year.
        Homeowner's insurance costs \$948 per year and property taxes are \$2370 per year.
        \begin{enumerate}
          \item Determine their monthly mortgage payment (principal and interest). \solution{\$960.18}
                \vfill
          \item Determine their monthly PMI payment. \solution{\$57.93}
                \vfill
          \item Determine their complete monthly payment (principal, interest, taxes, and insurance). \solution{\$1,294.61}
                \vfill
        \end{enumerate}
\clearpage

    \item Matthias and Joanna Schmitz are interested in buying a house selling for \$249,000. 
          The insurance and property taxes on the property are \$1380 and \$1980 per year, respectively. 
          The Schmitzes' bank requires 
              a 15\% down payment
              and a payment of 2 points at closing.
          \begin{enumerate}\setlength{\itemsep}{1in}
             \item What would the Schmitzes' down payment be? \solution{\$37,350.00}
                  \vfill
             \item What is the mortgage amount? \solution{\$211,650.00}
                  \vfill
             \item Determine the closing costs (down payment and points). \solution{\$41,583.00}
                  \vfill
             \item If the Schmitzes want a 30 year mortgage at 7\% interest compounded monthly, 
                   determine the monthly mortgage payment (principal and interest). \solution{\$1,408.11}
                  \vfill
             \item Determine the Schmitzes' complete monthly payment (principal, interest, taxes, and insurance). \solution{\$1,688.11}
                  \vfill
          \end{enumerate}
\end{enumerate}

%</WORKSHEETS>
%%%%%%%%%%%%%%%%%%%%%%%%%%%%%%%%%%%%%%%%%%%%%%%%%%%%%%%%%%%%%%%%%%%%%%%%%%%%%%%%%%%%%%%%%%%%%%%%%%%%%%
%<*HWHEADER>
\HOMEWORK
%</HWHEADER>

%<*HOMEWORK>


\begin{Fenumerate}

      
  \item Frank and Lacey Capricorn are buying a \$130,000 house.
        They can afford an 8\% down payment, but the bank will charge them 0.7\% PMI per year.
        They will get a 20-year mortgage at 4.5\% interest.
        Home insurance will cost them \$1260 annually,
        and property taxes run \$1986 per year.
        How much will their total monthly housing payment (PITI) be?
        \solution*{%
          Their loan amount is \$119,600,
          so their mortgage payment (P\&I) will be \$756.65 per month.
          Taxes are \$165.50 per month.
          Homeowner's insurance is \$105 per month, and the PMI is \$69.77 per month,
          for a total monthly payment of \fbox{\$1096.92.}
        }\vfill \vfill

  \item Erik and Kristin Halvorson want to buy a house for \$148,000.
        They want a standard 30-year, fixed-rate mortgage.
        They can get an interest rate of 3.5\% from their bank,
        but they will have to pay two points to the bank.
        They have saved up enough for a 10\% down payment.
        Home insurance will cost \$1500 per year,
        and property taxes are \$1800 annually.
        \begin{enumerate}
          \item Calculate the Halvorsons' closing costs.
                \ifsolns
                  \par\soln
                  The closing costs consist of down payments and points and fees.
                  The down payment is 10\% of \$148,000, namely \$14,800.
                  Thus the loan amount is $\$148{,}000-\$14{,}800 = \$133{,}200$.
                  Two points are 2\% of the loan amount, namely
                  \$2664.
                  Thus their closing costs are
                 \\ \centering{$ \displaystyle \$14{,}800 + \$2{,}664 = \boxed{\$17{,}464.00.}$}
                \fi
                \studentsoln{\$17,464.00}\vfill
          \item Calculate the Halvorsons'
                total monthly housing payment (PITI).
                \ifsolns
                  \par\soln
                  Principal and Interest can be found from the present value formula.
                  We are looking for the $P$ such that
                  $\$133{,}200 = P \dfrac{1-(1+\frac{.035}{12})^{-12\cdot 30}}{\frac{.035}{12}}.$
                  Solving, we get $P=\$598.13$ per month for Principal and Interest.
                  
                  As for Taxes, they are \$1800 per year so $\$1800\div 12 = \$150$ per month.
                  Likewise the Insurance costs $\$1500 \div 12 = \$125$ per month.
                  We conclude that
                 \\ \centering{$ \displaystyle PITI = \$598.13 + \$150 + \$125 = \boxed{\$873.13\mbox{~per month.}}$}
                \fi
                \studentsoln{\$873.13}\vfill
        \end{enumerate}
\hwnewpage

  \item Marcus and Julia Quackenthorpe want to buy a \$175,000 house.
        They will put down a 20\% down payment
        and get a 25-year fixed-rate mortgage.
        The bank offers a 5\% interest rate,
        but if they pay one point,
        the bank will lower the interest rate to 4.5\%.
        \begin{enumerate}
          \item How much money will the Quackenthorpes borrow?
                \ifsolns \par\soln \fbox{\$140,000.} \fi\vfill
          \item If they go with the 5\% rate, how much will their monthly mortgage payment be?
                \ifsolns \par\soln \fbox{\$818.43.} \fi\vfill
                \studentsoln{\$818.43}
          \item Using that 5\% rate, how much will they pay for their house in total, over the 25 years?
                \ifsolns \par\soln $\$818.43 \cdot 12\cdot 25 + \$35,000 = \boxed{\$280{,}529;}$ 
                         accept \$245,529 for less credit (that's forgetting the down payment).\fi
                \studentsoln{\$280,529}\vfill
          \item If they pay one point and get a 4.5\% interest rate, how much will their monthly mortgage payment be?
                \ifsolns \par\soln \fbox{\$778.17.} \fi\vfill
                %\studentsoln{\$778.17}
          \item Using that 4.5\% rate, how much will they pay for their house in total, over the 25 years,
                \emph{including the one point they paid for the discounted rate}?
                \ifsolns \par\soln $\$778.17 \cdot 12\cdot 25 + \$1400 + \$35,000 = \boxed{\$269,851;}$ 
                accept \$234,851 for less credit (that's forgetting the down payment).\fi\vfill
                %\studentsoln{\$269,851}
          \item Should the Quackenthorpes pay for the discount or not?
                \ifsolns \par\soln \fbox{Yes, they should;} it will save them over \$10,000.\fi\vfill
        \end{enumerate}


  \item You are shopping for a new house.  You have saved up \$10,500
        towards a down payment, and you calculate that you can afford \$1100 per month
        for housing total (PITI).
        Real estate taxes for a house in your city will probably be about \$2300 per year,
        and insurance will be about \$1600 annually.
        You can get a 30-year mortgage with 4\% interest, compounded monthly.
        What is the most expensive house you could buy and still keep under your \$1100/month budget?
        \ifsolns
          \par\soln
          Your Taxes will be $\$2300\div 12 = \$191.67$ per month, and your Insurance will be
          $\$1600 \div 12 = \$133.33$.  That leaves
         \\ \centering{$ \displaystyle \$1100 - \$191.67 - \$133.33 = \$775.00$}
          per month for Principal and Interest.
          Thus you can afford a loan of 
         \\ \centering{$ \displaystyle PV = \$775 \dfrac{1-(1+\frac{.04}{12})^{-360}}{\frac{.04}{12}}=\$162{,}332.46.$}
          Adding in the \$10,500 down payment you have saved,
          you can afford a house costing
          $\$162{,}332.46 + \$10{,}500 = \boxed{\$172{,}832.46.}$
        \fi
        \studentsoln{\$172,832.46}\vfill

\end{Fenumerate} \ENDHOMEWORK
%%%%%%%%%%%%%%%%%%%%%%%%%%%%%%%%%%%%%%%%%%%%%%%%%%%%%%%%%%%%%%%%%%%%%%%%%%%%%%%%%%%%%%%%%%%%%%%%%%%%%%%

%\input{TaxBrackets}

%</HOMEWORK>

%%%%%%%%%%%%%  %<*WORKSHEETS>
%%%%%%%%%%%%%
%%%%%%%%%%%%%  \clearpage
%%%%%%%%%%%%%  \section{Income Tax, Part 1}
%%%%%%%%%%%%%
%%%%%%%%%%%%%
%%%%%%%%%%%%%
%%%%%%%%%%%%%  \begin{enumerate}
%%%%%%%%%%%%%    \item The difference between \defnstyle{deductions} and \defnstyle{credits} is\ldots \fillwithlines{\stretch{1}}
%%%%%%%%%%%%%    \item You are a single college student.
%%%%%%%%%%%%%          Your taxable income is \$7,500, and the government taxes your income at a tax rate of 10\%.
%%%%%%%%%%%%%          \begin{enumerate}
%%%%%%%%%%%%%            \item Find how much tax you will pay.
%%%%%%%%%%%%%                  \vfill
%%%%%%%%%%%%%            \item If you could claim a \$500 tax deduction for tuition, how much tax would you pay?
%%%%%%%%%%%%%                  \vfill
%%%%%%%%%%%%%            \item If you could claim a \$500 tax credit for tuition, how much tax would you pay?
%%%%%%%%%%%%%                  \vfill
%%%%%%%%%%%%%            \item Which is better for you, a deduction or a credit?
%%%%%%%%%%%%%                  \vfill
%%%%%%%%%%%%%          \end{enumerate}
%%%%%%%%%%%%%    \clearpage
%%%%%%%%%%%%%
%%%%%%%%%%%%%         \begin{tabular}{|ll|}\hline
%%%%%%%%%%%%%           Status & 2012 Standard Deduction \\ \hline
%%%%%%%%%%%%%           Married Filing Jointly    & \$11,900 \\
%%%%%%%%%%%%%           Head of Household         & \$8,700  \\
%%%%%%%%%%%%%           Single                    & \$5,950  \\
%%%%%%%%%%%%%           Married Filing Separately & \$5,950  \\ \hline
%%%%%%%%%%%%%         \end{tabular}
%%%%%%%%%%%%%
%%%%%%%%%%%%%         \fbox{2012 Exemption: \$3,800 per person}
%%%%%%%%%%%%%
%%%%%%%%%%%%%
%%%%%%%%%%%%%    \item Christine is a single woman without children who makes \$38,500 annually.  
%%%%%%%%%%%%%          Find her taxable income for 2012.
%%%%%%%%%%%%%          \vfill
%%%%%%%%%%%%%    \item Zeke and Susan are married and file their taxes jointly.
%%%%%%%%%%%%%          Zeke earns \$57,300 in wages as a city planner,
%%%%%%%%%%%%%          while Susan is a part-time teacher and earns \$12,450 per year.
%%%%%%%%%%%%%          They have three young children.
%%%%%%%%%%%%%          Find their taxable income for 2012.
%%%%%%%%%%%%%          \vfill
%%%%%%%%%%%%%    %\clearpage
%%%%%%%%%%%%%    \item The difference between \defnstyle{refundable} and \defnstyle{non-refundable credits} is \ldots \fillwithlines{\stretch{1}}
%%%%%%%%%%%%%    \item Abednego is a student filling out his tax return.
%%%%%%%%%%%%%          After taking his deductions and figuring his tax,
%%%%%%%%%%%%%          his tax initially comes to 
%%%%%%%%%%%%%          \$1732.
%%%%%%%%%%%%%          He gets a non-refundable tuition credit of \$2000
%%%%%%%%%%%%%          and a refundable Earned Income Credit of \$264.
%%%%%%%%%%%%%          How much will the government pay Abednego this year?
%%%%%%%%%%%%%          \vfill
%%%%%%%%%%%%%  %\clearpage
%%%%%%%%%%%%%  %  \item How taxation works:
%%%%%%%%%%%%%  \clearpage
%%%%%%%%%%%%%         \hbox{\hskip 0.0pt minus 1.0fil
%%%%%%%%%%%%%         {\small
%%%%%%%%%%%%%         \begin{tabular}{|lllll|} \hline
%%%%%%%%%%%%%         2012 
%%%%%%%%%%%%%          & Married 
%%%%%%%%%%%%%          & 
%%%%%%%%%%%%%          & 
%%%%%%%%%%%%%          & Married \\
%%%%%%%%%%%%%         Tax Rate
%%%%%%%%%%%%%          & Filing Jointly 
%%%%%%%%%%%%%          & Head of Household
%%%%%%%%%%%%%          & Single 
%%%%%%%%%%%%%          & Filing Separately \\ \hline
%%%%%%%%%%%%%         10\% & Not over \$17,400        & Not over \$12,400      & Not over \$8,700       & Not over \$8,700       \\
%%%%%%%%%%%%%         15\% &  \$17,400 ? \$70,700    & \$12,400 -- \$47,350   & \$8,700 ? \$35,350    & \$8,700 ? \$35,350    \\
%%%%%%%%%%%%%         25\% &  \$70,700 ? \$142,700   & \$47,350 -- \$122,300  & \$35,350 ? \$85,650   & \$35,350 ? \$71,350   \\
%%%%%%%%%%%%%         28\% & \$142,700 ? \$217,450   & \$122,300 -- \$198,050 & \$85,650 ? \$178,650  & \$71,350 ? \$108,725  \\
%%%%%%%%%%%%%         33\% & \$217,450 ? \$388,350   & \$198,050 -- \$388,350 & \$178,650 -?\$388,350 & \$108,725 -?\$194,175 \\
%%%%%%%%%%%%%         35\% & Over \$388,350           & Over \$388,350         & Over \$388,350         & Over \$194,175         \\ \hline
%%%%%%%%%%%%%         \end{tabular}}\hfil}
%%%%%%%%%%%%%
%%%%%%%%%%%%%    \item John and Martha Kent are a married couple with a 2012 taxable income of \$53,000.
%%%%%%%%%%%%%          How much is their 2012 tax (before any credits are applied)?
%%%%%%%%%%%%%          \vfill
%%%%%%%%%%%%%    \item Susan Smith files her taxes as Head of Household.  Her taxable income was \$42,000.
%%%%%%%%%%%%%          How much is her 2012 tax (before any credits are applied)?
%%%%%%%%%%%%%          \vfill
%%%%%%%%%%%%%    \item Shadrach Goldberg is a single workaholic with a 2010 taxable income of \$127,000.
%%%%%%%%%%%%%          How much was his 2012 tax (before any credits were applied)?
%%%%%%%%%%%%%          \vfill
%%%%%%%%%%%%%  \clearpage
%%%%%%%%%%%%%    \item Archibald and Alexandra Campbell are married but don't have any children yet.
%%%%%%%%%%%%%          In 2012 Archibald's income at the meatpacking plant was \$53,210.37,
%%%%%%%%%%%%%          while Alexandra earned wages of \$22,329.53 as a waitress.
%%%%%%%%%%%%%          At the bank they earned taxable interest of \$121.83.
%%%%%%%%%%%%%          According to their W-2's, Archibald had \$6213.20 withheld from his paychecks throughout the year,
%%%%%%%%%%%%%          while Alexandra had \$1923.10 withheld.
%%%%%%%%%%%%%          
%%%%%%%%%%%%%          Fill out Form 1040EZ for the Campbells.  (Be sure to fill out the worksheet for line 8.)
%%%%%%%%%%%%%          
%%%%%%%%%%%%%          \emph{(For this and all other tax problems, you may assume that I have given you all relevant information.
%%%%%%%%%%%%%          For example, because I didn't say anything about a nontaxable combat pay election,
%%%%%%%%%%%%%          you may assume the Campbells don't have anything for that line of the form.)}
%%%%%%%%%%%%%          
%%%%%%%%%%%%%  \end{enumerate}
%%%%%%%%%%%%%
%%%%%%%%%%%%%  %</WORKSHEETS>
%%%%%%%%%%%%%
%%%%%%%%%%%%%  %<*HWHEADER>
%%%%%%%%%%%%%  \HOMEWORK
%%%%%%%%%%%%%
%%%%%%%%%%%%%  \begin{center}
%%%%%%%%%%%%%    \fbox{\begin{minipage}{5in}
%%%%%%%%%%%%%      \begin{center}
%%%%%%%%%%%%%        \bfseries Instructions for Income Tax Homework
%%%%%%%%%%%%%      \end{center}
%%%%%%%%%%%%%      \begin{itemize}\setlength{\itemsep}{0pt}
%%%%%%%%%%%%%        \item In the following questions, you may assume that I have given you all relevant information;
%%%%%%%%%%%%%              for example, if I do not tell you that a person has a farm, then that person's 
%%%%%%%%%%%%%              ``farm income (or loss)'' (Form 1040, line 18) will be zero.
%%%%%%%%%%%%%
%%%%%%%%%%%%%        \item Some questions ask you to fill out tax forms,
%%%%%%%%%%%%%              which may be downloaded from the course website at
%%%%%%%%%%%%%              \begin{center}
%%%%%%%%%%%%%                  \url{http://home.snc.edu/anders/hendrickson/123/taxforms/}.
%%%%%%%%%%%%%              \end{center}
%%%%%%%%%%%%%              Fill out separate tax forms for each question.
%%%%%%%%%%%%%        \item When computing the tax, be sure to use the 2012 Tax Tables, which are also found on that webpage;
%%%%%%%%%%%%%              I recommend viewing the Tax Tables online rather than printing them out,
%%%%%%%%%%%%%              so as to save paper.
%%%%%%%%%%%%%        \item All married couples will file jointly.
%%%%%%%%%%%%%              No filer can be claimed as a dependent on his or her parents' return.
%%%%%%%%%%%%%        \item When an entry on the tax form should be zero because it's irrelevant,
%%%%%%%%%%%%%              you may leave it blank instead.
%%%%%%%%%%%%%        \item None of these examples qualifies for the Earned Income Credit.
%%%%%%%%%%%%%        \item For the personal information at the top of the tax form
%%%%%%%%%%%%%              (e.g.~Social Security numbers and addresses)
%%%%%%%%%%%%%              please make up the answers yourself.
%%%%%%%%%%%%%              You may sign your own name as the ``Paid Preparer'' at the bottom of the tax return.
%%%%%%%%%%%%%        %\item Be sure to reach the ultimate answer---either how much the person will get as a refund,
%%%%%%%%%%%%%        %      or how much the person still owes.
%%%%%%%%%%%%%      \end{itemize}
%%%%%%%%%%%%%    \end{minipage}}
%%%%%%%%%%%%%  \end{center}
%%%%%%%%%%%%%
%%%%%%%%%%%%%
%%%%%%%%%%%%%  \clearpage
%%%%%%%%%%%%%  %</HWHEADER>
%%%%%%%%%%%%%
%%%%%%%%%%%%%  %<*HOMEWORK>
%%%%%%%%%%%%%
%%%%%%%%%%%%%  \def\turnin#1{\fbox{{\sc Turn in:} #1}}
%%%%%%%%%%%%%  \begin{Fenumerate}
%%%%%%%%%%%%%
%%%%%%%%%%%%%
%%%%%%%%%%%%%  \item The IRS has announced that the \emph{2011} marginal income tax brackets will be
%%%%%%%%%%%%%
%%%%%%%%%%%%%         \hbox{\hskip 0.0pt plus 1.0fil minus 1.0fil
%%%%%%%%%%%%%         {\footnotesize
%%%%%%%%%%%%%         \begin{tabular}{|lllll|} \hline
%%%%%%%%%%%%%         Tax Rate
%%%%%%%%%%%%%          & Married Filing Jointly 
%%%%%%%%%%%%%          & Head of Household
%%%%%%%%%%%%%          & Single 
%%%%%%%%%%%%%          & Married Filing Separately \\
%%%%%%%%%%%%%         10\% & Not over \$17,000        & Not over \$12,150      & Not over \$8,500       & Not over \$8,500       \\
%%%%%%%%%%%%%         15\% &  \$17,000 -- \$69,000    & \$12,150 --  \$46,250   & \$8,500 -- \$34,500    & \$8,500-- \$34,500    \\
%%%%%%%%%%%%%         25\% &  \$69,000 -- \$139,350   & \$46,250 --  \$119,400  & \$34,500 -- \$83,600   & \$34,500 -- \$69,675   \\
%%%%%%%%%%%%%         28\% & \$139,350 -- \$212,300   & \$119,400 --  \$193,350 & \$83,600 -- \$174,400  & \$69,675 -- \$106,150  \\
%%%%%%%%%%%%%         33\% & \$212,300 -- \$379,150   & \$193,350 --  \$379,150 & \$174,400 -- \$379,150 & \$106,150 --  \$189,575 \\
%%%%%%%%%%%%%         35\% & Over \$379,150           & Over \$379,150         & Over \$379,150         & Over \$189,575         \\ \hline
%%%%%%%%%%%%%         \end{tabular}}\hfil}
%%%%%%%%%%%%%
%%%%%%%%%%%%%         \begin{tabular}{|ll|}\hline
%%%%%%%%%%%%%           Status & 2011 Standard Deduction \\
%%%%%%%%%%%%%           Married Filing Jointly    & \$11,600 \\
%%%%%%%%%%%%%           Head of Household         & \$8,500  \\
%%%%%%%%%%%%%           Single                    & \$5,800  \\
%%%%%%%%%%%%%           Married Filing Separately & \$5,800  \\ \hline
%%%%%%%%%%%%%         \end{tabular}
%%%%%%%%%%%%%
%%%%%%%%%%%%%         \fbox{2011 Exemption: \$3,700}
%%%%%%%%%%%%%
%%%%%%%%%%%%%                 
%%%%%%%%%%%%%         \label{prob:Zeke} 
%%%%%%%%%%%%%         \begin{enumerate}
%%%%%%%%%%%%%           \item Zeke wants to estimate how much he will have to pay on his 2011 tax return (which will be filed April 2012).
%%%%%%%%%%%%%                 Zeke is married with three young children,
%%%%%%%%%%%%%                 and his adjusted gross income is \$53,000.
%%%%%%%%%%%%%                 He will take the standard deduction.
%%%%%%%%%%%%%                 Use the tables above to estimate as accurately as possible how much Zeke's taxes will be,
%%%%%%%%%%%%%                 \emph{before} any credits are applied.
%%%%%%%%%%%%%                 
%%%%%%%%%%%%%                 \ifsolns
%%%%%%%%%%%%%                   \soln
%%%%%%%%%%%%%                   The standard deduction is \$11,600, and they get five exemptions, so the taxable income is
%%%%%%%%%%%%%                  \\ \centering{$ \displaystyle 53000-11600-5\cdot 3700 = 22900.$}
%%%%%%%%%%%%%                   The first \$17000 is taxed at 10\%, so that yields $\$17000\times 0.10=\$1700$ in tax. \\
%%%%%%%%%%%%%                   The remainder, $22900-17000=5900$, is taxed at 15\%, so that yields $\$5900\times 0.15 = 885.$
%%%%%%%%%%%%%                   Thus the total tax will be
%%%%%%%%%%%%%                  \\ \centering{$ \displaystyle 1700 + 885 = \boxed{\$2585.00.}$}
%%%%%%%%%%%%%                 \fi
%%%%%%%%%%%%%           \item Money is tight these days, so Zeke's wife Susanna is thinking about getting a job to bring in extra income.
%%%%%%%%%%%%%                 She can earn \$20 per hour working in the hospital.
%%%%%%%%%%%%%                 The social security tax takes 4.2\% of her paycheck, and the Medicare tax takes 1.45\%.
%%%%%%%%%%%%%                 They would have to pay \$6/hour for child care while Susanna is at work.
%%%%%%%%%%%%%                 After social security, Medicare, income taxes, and child care,
%%%%%%%%%%%%%                 how much extra money is Susanna bringing in per hour?
%%%%%%%%%%%%%                 
%%%%%%%%%%%%%                 \ifsolns
%%%%%%%%%%%%%                   \soln
%%%%%%%%%%%%%                   Every hour she works she earns \$20.
%%%%%%%%%%%%%                   Every hour she works, she also has to pay the following:
%%%%%%%%%%%%%                   
%%%%%%%%%%%%%                   $\begin{array}{l@{{}={}}rll}
%%%%%%%%%%%%%                     \$20 \times 0.042   & \$0.84 & \mbox{for social security} \\
%%%%%%%%%%%%%                     \$20 \times 0.0145  & \$0.29 & \mbox{for Medicare} \\
%%%%%%%%%%%%%                     \$20 \times 0.15    & \$3.00 & \mbox{for income taxes} & \mbox{(since they're in the 15\% bracket)} \\
%%%%%%%%%%%%%                     \multicolumn{1}{l}{}& \$6.00 & \mbox{for child care} \\ \cline{1-3}
%%%%%%%%%%%%%                     \multicolumn{1}{l}{}& \$10.13& \mbox{total}
%%%%%%%%%%%%%                   \end{array}$
%%%%%%%%%%%%%                   Thus she is actually earning only $\$20.00 - \$10.13 = {}$\fbox{\$9.87 per hour.}
%%%%%%%%%%%%%                 \fi
%%%%%%%%%%%%%         \end{enumerate}
%%%%%%%%%%%%%
%%%%%%%%%%%%%  \item For each of the following people, fill out a 2010 tax return,
%%%%%%%%%%%%%        \underline{using form 1040EZ}.
%%%%%%%%%%%%%        
%%%%%%%%%%%%%        \begin{enumerate}
%%%%%%%%%%%%%          \item Shadrach Cohen is single; because his tax situation is utterly simple, he chooses to file form 1040EZ.
%%%%%%%%%%%%%                His W-2 from work reports \$38,127.28 in income and \$3210.15 in federal withholdings.
%%%%%%%%%%%%%                His bank's 1099-INT reports \$57.32 in taxable interest.
%%%%%%%%%%%%%                \turnin{Form 1040EZ, including the back side.}
%%%%%%%%%%%%%                \ifsolns
%%%%%%%%%%%%%                  \par\soln See attached 1040EZ.
%%%%%%%%%%%%%                \fi
%%%%%%%%%%%%%          \item Nick and Jessica Cointreau are married.
%%%%%%%%%%%%%                Nick's W-2 from the city sanitation department reports \$42,329.72 in income and \$4121.32 in federal withholdings.
%%%%%%%%%%%%%                Jessica's office sent a W-2 reporting \$31,573.09 in income and \$2917.31 in withholdings.
%%%%%%%%%%%%%                Their bank's 1099-INT reports \$181.79 in taxable interest. \\
%%%%%%%%%%%%%                \turnin{Form 1040EZ, including the back side.}
%%%%%%%%%%%%%                \ifsolns
%%%%%%%%%%%%%                  \par\soln See attached 1040EZ.
%%%%%%%%%%%%%                \fi
%%%%%%%%%%%%%        \end{enumerate}
%%%%%%%%%%%%%
%%%%%%%%%%%%%
%%%%%%%%%%%%%  \end{Fenumerate} \ENDHOMEWORK
%%%%%%%%%%%%%
%%%%%%%%%%%%%
%%%%%%%%%%%%%  %</HOMEWORK>
%%%%%%%%%%%%%
%%%%%%%%%%%%%  %<*WORKSHEETS>
%%%%%%%%%%%%%
%%%%%%%%%%%%%
%%%%%%%%%%%%%
%%%%%%%%%%%%%
%%%%%%%%%%%%%
%%%%%%%%%%%%%  \clearpage
%%%%%%%%%%%%%  \section{Income Tax, Part 2}
%%%%%%%%%%%%%
%%%%%%%%%%%%%
%%%%%%%%%%%%%  \begin{enumerate}
%%%%%%%%%%%%%    
%%%%%%%%%%%%%    %Itemizing deductions
%%%%%%%%%%%%%    \item Joseph and Bernadette Zwingling are married and file jointly.
%%%%%%%%%%%%%          Their Adjusted Gross Income (Form 1040, line 38) for 2010 is \$63,210.32.
%%%%%%%%%%%%%          They paid \$6,532.10 in mortgage interest on their home, as well as \$2,103.23 in real estate taxes.
%%%%%%%%%%%%%          They gave \$3,320.00 to their church and \$2,100.00 to other charities,
%%%%%%%%%%%%%          as well as donating a used computer worth \$200.00 to the Goodwill Store.
%%%%%%%%%%%%%          Joseph's knee surgery cost them \$4,832.00, and he had to spend \$1350.86 for job travel.
%%%%%%%%%%%%%          \begin{enumerate}
%%%%%%%%%%%%%            \item Fill out Schedule A for the Zwinglings.
%%%%%%%%%%%%%                  \fillwithlines{\stretch{1}}
%%%%%%%%%%%%%            \item The standard deduction for a married couple filing jointly is \$11,400.
%%%%%%%%%%%%%                  Should they itemize or take the standard deduction?
%%%%%%%%%%%%%                  \vfill
%%%%%%%%%%%%%            \item The Zwinglings are in the 15\% tax bracket.  How much will your decision from part (b) reduce their tax bill?
%%%%%%%%%%%%%                  \vfill
%%%%%%%%%%%%%          \end{enumerate}
%%%%%%%%%%%%%
%%%%%%%%%%%%%    \clearpage
%%%%%%%%%%%%%    %Tuition and fee deductions
%%%%%%%%%%%%%    \item Giuseppe Garibaldi of Sleepy Ear, MN finished college years ago, 
%%%%%%%%%%%%%          but he took a business course
%%%%%%%%%%%%%          at the local community college in 2010.  He paid \$5320 in tuition.
%%%%%%%%%%%%%          He is in the 25\% marginal tax bracket.
%%%%%%%%%%%%%          \begin{enumerate}
%%%%%%%%%%%%%            \item Is he eligible for the tuition and fees deduction?  If so, how much will it reduce his taxes?
%%%%%%%%%%%%%                  \vfill
%%%%%%%%%%%%%            \item Is he eligible for the American Opportunity Credit?  If so, how much would it be worth?
%%%%%%%%%%%%%                  \vfill
%%%%%%%%%%%%%            \item Is he eligible for the Lifetime Learning Credit?  If so, how much would it be worth?
%%%%%%%%%%%%%                  \vfill
%%%%%%%%%%%%%          \end{enumerate}
%%%%%%%%%%%%%
%%%%%%%%%%%%%    \item Wolfgang Schmitz is a full-time student working towards a B.A. in brewing at East Dakota State University.
%%%%%%%%%%%%%          He paid \$12,297 in tuition and fees in 2010.  He is in the 15\% marginal tax bracket.
%%%%%%%%%%%%%          \begin{enumerate}
%%%%%%%%%%%%%            \item Is he eligible for the tuition and fees deduction?  If so, how much will it reduce his taxes?
%%%%%%%%%%%%%                  \vfill
%%%%%%%%%%%%%            \item Is he eligible for the American Opportunity Credit?  If so, how much would it be worth?
%%%%%%%%%%%%%                  \vfill
%%%%%%%%%%%%%            \item Is he eligible for the Lifetime Learning Credit?  If so, how much would it be worth?
%%%%%%%%%%%%%                  \vfill
%%%%%%%%%%%%%          \end{enumerate}
%%%%%%%%%%%%%
%%%%%%%%%%%%%    \item Lars Olson of New Trondheim, ND is a college graduate and farmer.
%%%%%%%%%%%%%          He took a Norwegian language course in 2010 from Bergen Lutheran College,
%%%%%%%%%%%%%          which cost \$4100 in tuition.
%%%%%%%%%%%%%          He is in the 25\% marginal tax bracket.
%%%%%%%%%%%%%          \begin{enumerate}
%%%%%%%%%%%%%            \item Is he eligible for the tuition and fees deduction?  If so, how much will it reduce his taxes?
%%%%%%%%%%%%%                  \vfill
%%%%%%%%%%%%%            \item Is he eligible for the American Opportunity Credit?  If so, how much would it be worth?
%%%%%%%%%%%%%                  \vfill
%%%%%%%%%%%%%            \item Is he eligible for the Lifetime Learning Credit?  If so, how much would it be worth?
%%%%%%%%%%%%%                  \vfill
%%%%%%%%%%%%%          \end{enumerate}
%%%%%%%%%%%%%    \vspace{-1in}
%%%%%%%%%%%%%
%%%%%%%%%%%%%  \end{enumerate}
%%%%%%%%%%%%%
%%%%%%%%%%%%%
%%%%%%%%%%%%%  %</WORKSHEETS>
%%%%%%%%%%%%%
%%%%%%%%%%%%%  %<*HWHEADER>
%%%%%%%%%%%%%  \HOMEWORK
%%%%%%%%%%%%%  %</HWHEADER>
%%%%%%%%%%%%%
%%%%%%%%%%%%%  %<*HOMEWORK>
%%%%%%%%%%%%%
%%%%%%%%%%%%%         \hbox{\hskip 0.0pt minus 1.0fil
%%%%%%%%%%%%%         {\small
%%%%%%%%%%%%%         \begin{tabular}{|lllll|} \hline
%%%%%%%%%%%%%         2010 
%%%%%%%%%%%%%          & Married 
%%%%%%%%%%%%%          & 
%%%%%%%%%%%%%          & 
%%%%%%%%%%%%%          & Married \\
%%%%%%%%%%%%%         Tax Rate
%%%%%%%%%%%%%          & Filing Jointly 
%%%%%%%%%%%%%          & Head of Household
%%%%%%%%%%%%%          & Single 
%%%%%%%%%%%%%          & Filing Separately \\ \hline
%%%%%%%%%%%%%         10\% & Not over \$16,750        & Not over \$11,950      & Not over \$8,375       & Not over \$8,375       \\
%%%%%%%%%%%%%         15\% &  \$16,750 ? \$68,000    & \$11,950 -- \$45,550   & \$8,375 ? \$34,000    & \$8,375 ? \$34,000    \\
%%%%%%%%%%%%%         25\% &  \$68,000 ? \$137,300   & \$45,550 -- \$117,650  & \$34,000 ? \$82,400   & \$34,000 ? \$68,650   \\
%%%%%%%%%%%%%         28\% & \$137,300 ? \$209,250   & \$117,650 -- \$190,550 & \$82,400 ? \$171,850  & \$68,650 ? \$104,625  \\
%%%%%%%%%%%%%         33\% & \$209,250 ? \$373,650   & \$190,550 -- \$373,650 & \$171,850 -?\$373,650 & \$104,625 -- \$186,825 \\
%%%%%%%%%%%%%         35\% & Over \$373,650           & Over \$373,650         & Over \$373,650         & Over \$186,825         \\ \hline
%%%%%%%%%%%%%         \end{tabular}}\hfil}
%%%%%%%%%%%%%
%%%%%%%%%%%%%         \begin{tabular}{|ll|}\hline
%%%%%%%%%%%%%           Status & 2010 Standard Deduction \\ \hline
%%%%%%%%%%%%%           Married Filing Jointly    & \$11,400 \\
%%%%%%%%%%%%%           Head of Household         & \$8,400  \\
%%%%%%%%%%%%%           Single                    & \$5,700  \\
%%%%%%%%%%%%%           Married Filing Separately & \$5,700  \\ \hline
%%%%%%%%%%%%%         \end{tabular}
%%%%%%%%%%%%%
%%%%%%%%%%%%%  \begin{Fenumerate}
%%%%%%%%%%%%%
%%%%%%%%%%%%%  \item Thomas and Susan Ogilvie 
%%%%%%%%%%%%%        gave \$3725 to charities in 2010,
%%%%%%%%%%%%%        and they also donated \$125 worth of used items to the St.~Vincent de Paul thrift store.
%%%%%%%%%%%%%        Susan is still recovering from a kidney transplant, 
%%%%%%%%%%%%%        and they had \$4,521.39 in medical expenses.
%%%%%%%%%%%%%        Thomas had \$853.19 of unreimbursed expenses for his job as a building inspector.
%%%%%%%%%%%%%        Their form 1098 reports that they paid \$5,760.23 in home mortgage interest,
%%%%%%%%%%%%%        and their property tax bill was \$2012.00.
%%%%%%%%%%%%%        On Form 1040, their Adjusted Gross Income on line 38 is \$53,240.
%%%%%%%%%%%%%        They lie in the 15\% marginal tax bracket.
%%%%%%%%%%%%%        \begin{enumerate}
%%%%%%%%%%%%%          \item Fill out Schedule A for the Ogilvies.  \turnin{Schedule A}
%%%%%%%%%%%%%                \ifsolns
%%%%%%%%%%%%%                  \par\soln See attached Schedule A.
%%%%%%%%%%%%%                \fi
%%%%%%%%%%%%%          \item Should they itemize deductions or take the standard deduction?                 
%%%%%%%%%%%%%                \ifsolns
%%%%%%%%%%%%%                  \par\soln If they itemize, they can deduct \$12,150.62.
%%%%%%%%%%%%%                  The standard deduction for 2010 is only \$11,400.00.
%%%%%%%%%%%%%                  \fbox{Thus they should itemize deductions.}
%%%%%%%%%%%%%                \fi
%%%%%%%%%%%%%          \item How much tax will that choice save them?
%%%%%%%%%%%%%                \ifsolns
%%%%%%%%%%%%%                  \par\soln It will save them $\$12150.62-\$11400.00 = \boxed{\$750.62.}$
%%%%%%%%%%%%%                \fi
%%%%%%%%%%%%%        \end{enumerate}
%%%%%%%%%%%%%
%%%%%%%%%%%%%
%%%%%%%%%%%%%
%%%%%%%%%%%%%
%%%%%%%%%%%%%  \item Fill out a 2010 tax return 
%%%%%%%%%%%%%        for the following couple
%%%%%%%%%%%%%        using Form 1040.
%%%%%%%%%%%%%        Be sure to pay attention to the Making Work Pay credit and the Child Tax Credit. \\
%%%%%%%%%%%%%        \turnin{Form 1040 and Schedule M}
%%%%%%%%%%%%%        
%%%%%%%%%%%%%        %All relevant tax forms can be downloaded from the course website at
%%%%%%%%%%%%%        %    \url{http://www.cord.edu/faculty/ahendric/105/taxforms/}.
%%%%%%%%%%%%%        %The 2010 Tax Tables are also found on that webpage;
%%%%%%%%%%%%%        %I recommend viewing the Tax Tables online rather than printing them out,
%%%%%%%%%%%%%        %so as to save paper.
%%%%%%%%%%%%%        
%%%%%%%%%%%%%        %\begin{enumerate}
%%%%%%%%%%%%%        %  \item 
%%%%%%%%%%%%%                Abednego and Hannah Leibowitz have one daughter, Jessica, who is 7 years old.
%%%%%%%%%%%%%                They moved to Moose Lake this year; according to Form 3903, their eligible moving expenses are \$873.12.
%%%%%%%%%%%%%                The W-2's report \$68,721.14 from Abednego's job at the steel mill, with \$8,731.21 in withholdings,
%%%%%%%%%%%%%                and \$23,421.38 from Hanna's part-time job at the junior high school, with \$412.18 in withholdings.
%%%%%%%%%%%%%                Moose Lake Credit Union has sent a 1099-INT showing \$91.23 in taxable interest.
%%%%%%%%%%%%%                Hannah is still paying off her student loans from college,
%%%%%%%%%%%%%                and paid \$1,210.13 in interest this year.
%%%%%%%%%%%%%                They will take the standard deduction.
%%%%%%%%%%%%%                
%%%%%%%%%%%%%                \ifsolns
%%%%%%%%%%%%%                  \par\soln See attached Form 1040 and Schedule M.
%%%%%%%%%%%%%                \fi
%%%%%%%%%%%%%        %\end{enumerate}
%%%%%%%%%%%%%        \pageover
%%%%%%%%%%%%%        
%%%%%%%%%%%%%    \item Sven Olson is a single man working his way through college.  
%%%%%%%%%%%%%          His W-2 reads \$18,243.12 in wages, and %this is also the amount on form 1040, line 22.
%%%%%%%%%%%%%          Sven paid \$8,892.00 in tuition and fees 
%%%%%%%%%%%%%          to St.~Hallvard's College this year.
%%%%%%%%%%%%%          %Sven will take the standard deduction.
%%%%%%%%%%%%%          
%%%%%%%%%%%%%          \enlargethispage{\baselineskip}
%%%%%%%%%%%%%          Sven wants to know how to use the tuition and fees to his best advantage.
%%%%%%%%%%%%%          Use Forms 8917 and 8863 to answer the following questions.
%%%%%%%%%%%%%          \emph{(You should complete form 8863 two separate times, once for part (c) and once for part (d).)}
%%%%%%%%%%%%%          \begin{enumerate}
%%%%%%%%%%%%%            \item What marginal tax bracket is Sven in?
%%%%%%%%%%%%%                  \ifsolns
%%%%%%%%%%%%%                    \par\soln He is in \fbox{the 15\% tax bracket.}
%%%%%%%%%%%%%                  \fi
%%%%%%%%%%%%%            \item If Sven takes the Tuition and Fees deduction (Form 8917, and Form 1040 line 34),
%%%%%%%%%%%%%                  how much will that reduce his tax?
%%%%%%%%%%%%%                  \turnin{Form 8917}
%%%%%%%%%%%%%                  \emph{(Form 1040, line 22 will just be Sven's wages.
%%%%%%%%%%%%%                         On Form 8917, enter \$0.00 on line 4.)}
%%%%%%%%%%%%%                  \ifsolns
%%%%%%%%%%%%%                    \par\soln See attached Form 8917.  His deduction is \$4000, so he would save
%%%%%%%%%%%%%                              $\$4000 \times 0.15 = \boxed{\$600.}$
%%%%%%%%%%%%%                  \fi
%%%%%%%%%%%%%            \item If Sven takes the American Opportunity Credit (Form 8863),
%%%%%%%%%%%%%                  how large would that credit be?
%%%%%%%%%%%%%                  How much of this credit is refundable, and how much nonrefundable?
%%%%%%%%%%%%%                  \turnin{Form 8863}
%%%%%%%%%%%%%                  \emph{(Form 1040, line 38 will just be Sven's wages.
%%%%%%%%%%%%%                         On Form 8863, enter the amount from line 15 on line 23.)}
%%%%%%%%%%%%%                  \ifsolns
%%%%%%%%%%%%%                    \par\soln See attached Form 8863.
%%%%%%%%%%%%%                              The credit will be \fbox{\$2500.00},
%%%%%%%%%%%%%                              of which \fbox{\$1000 is refundable and \$1500 is nonrefundable.}
%%%%%%%%%%%%%                  \fi
%%%%%%%%%%%%%            \item If Sven takes the Lifetime Learning Credit (Form 8863),
%%%%%%%%%%%%%                  how much will that reduce his tax?
%%%%%%%%%%%%%                  How much of this credit is refundable, and how much nonrefundable?
%%%%%%%%%%%%%                  \turnin{Form 8863}
%%%%%%%%%%%%%                  \emph{(Form 1040, line 38 will just be Sven's wages.
%%%%%%%%%%%%%                         On Form 8863, enter the amount from line 22 on line 23.)}
%%%%%%%%%%%%%                  \ifsolns
%%%%%%%%%%%%%                    \par\soln See attached Form 8863.
%%%%%%%%%%%%%                              The credit will be \fbox{\$1778.40},
%%%%%%%%%%%%%                              of which \fbox{\$0 is refundable and \$1778.40 is nonrefundable.}
%%%%%%%%%%%%%                  \fi
%%%%%%%%%%%%%          \end{enumerate}
%%%%%%%%%%%%%
%%%%%%%%%%%%%
%%%%%%%%%%%%%
%%%%%%%%%%%%%  \end{Fenumerate} \ENDHOMEWORK
%%%%%%%%%%%%%
%%%%%%%%%%%%%  %</HOMEWORK>

%<*WORKSHEETS>

\clearpage