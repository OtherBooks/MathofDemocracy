\section{Fair Division} \label{sec:FairDivision}
\begin{enumerate}

\item My friend and I decide have pizza for dinner. We order pizza which is half red sauce with pepperoni and half garlic sauce with ham and pineapple. To my friend, pizza is pizza. He has no preference for one type of pizza over another. On the other hand, I have an allergy to oregano and cannot eat any red sauce. It gives me a horrible stomach ache and I'm grumpy for about eight hours. 
\begin{enumerate}
	\item What is a fair way to divide this pizza so that my friend and I both believe we are getting a value of at least 50\% of the pizza. \vfill
	\item My friend never remembers that I do not eat red sauce. Therefore, he divides the pizza so that each half has exactly half red sauce and half garlic sauce. Am I able to choose a slice where I receive 50\% of the value of the pizza? \solution{Yes} \vfill
	\item Suppose instead, I slice the pizza. When I do so, because I value the red sauce part as nothing, I divide the pizza so that one slice is all garlic sauce and the other slice is all red sauce. Have I divided the pizza in such a way that each slice is worth 50\%  to me? \solution{No, because one slice is worth 100\% and the other 0\% to me.} \vfill
	\item Is there a way that I could divide the pizza in order to have each slice worth 50\% to me and have all the red sauce on one of the slices? \solution{Yes.  One slice would have all red plus 1/2 of the garlic.} \vfill
	\item In the case above, which of the slices with my friend choose? \solution{The larger one.}\vfill
\end{enumerate}

\clearpage
The basic issue at all fair division problems can be stated in a reasonably simple terms: How can something that must be shared by a set of competing parties be divided among them in a way that ensures that each party receives a fair share? Of course, part of the answer to this question will involve defining what we mean by a fair share.

We are going to start with the most common method for fair division, that of the divider-chooser. \index{divider-chooser}  This is the problem where there are two people and one thing to divide. One person has chosen to do the cutting, and then the other person gets to choose which part they want.
              	\fillwithlines{\stretch{1}}

\begin{description}
	\item[Rationality] \ifsolns Each of the players is a thinking, rational agent seeking to maximize his or her share of the booty $S$.  We further assume that in pursuit of this goal, a player's moves are based on reason alone. \fi\vfill
	\item[Cooperation] \ifsolns The players are willing participants and accept the rules of the game as binding.  The rules are such that after a \emph{finite} number of moves by the players the game terminates with a division of $S$. \fi              	\fillwithlines{\stretch{1}}

	\item[Privacy]\ifsolns Players have no useful information on the other players' value systems, and thus if what kinds of moves they are going to make in the game. \fi              	\fillwithlines{\stretch{1}}

	\item[Symmetry] \ifsolns Players have \emph{equal} rights in sharing the set $S$.  A consequence of this assumption is that at a minimum, each player is entitled to a \emph{proportional} share of $S$, that is $\frac{S}{N}$ when there are $N$ players. \fi               	\fillwithlines{\stretch{1}}

\end{description}


\item \boxedblank{\textbf{Fair Share:} \ifsolns Suppose that $s$ denotes a share of the booty $S$ and $P$ is one of the players in a fair division game with $N$ players.  We say that $s$ is a \textbf{fair share to player $P$} if $s$ is worth \textit{at least} $1/N$th of the total value of $S$ \textit{in the opinion of $P$}.\fi} 
              	\fillwithlines{\stretch{1}}
\index{fair share}
 
 \clearpage

\subsection{Divider-Chooser}
  \item \boxedblank[2in]{\textbf{Divider-Chooser:}\fillwithlines{\stretch{1}}}
              	
\item Sometimes it is easier to work with fair division problems when we put a monetary value on the object being divided because then we can quantify how much each piece is worth to each player. Suppose the cost of the pizza which is half red sauce and half white sauce is \$10. 
\begin{enumerate}
		\item My friend never remembers that I do not eat red sauce. Therefore, he divide the pizza so that each half has exactly half red sauce and half garlic sauce. How much is each slice of pizza worth to my friend? How much is each slice of pizza worth to me? \fillwithlines{\stretch{1}}
	\item Suppose instead, I slice the pizza. When I do so, because I value the red sauce part as nothing, I divide the pizza so that one slice is all garlic sauce and the other slice is all red sauce. How much is each slice of pizza worth to me? How much of each slice of pizza worth to my friend? If I do the division, why is this not a fair division of the pizza? What happens if my friend is the one doing the division? Is it fair?  \fillwithlines{\stretch{1}}
	\item Now, divide the pizza in such a way that all the red sauce is in one slice yet both slices are worth five dollars to me. How much is each slice of the pizza worth to my friend? If I do the division, why is this a fair division of the pizza? What happens if my friend is the one doing the division? Is it fair? \fillwithlines{\stretch{1}}
	
\end{enumerate}

\clearpage