%chapter-Voting.tex

%<*CHAPTERHEADER>

\declareproblemlettering{V}
\pagestyle{fancy}
\cleartooddpage


\chapter{Voting Theory}\label{ch:votingtheory}

%</CHAPTERHEADER>

%<*WORKSHEETS>
\section{Voting Systems}


\begin{enumerate}
  \item The Bellevue Taxicab Union is preparing to elect its president. 
        Three members are considering running for president:  Alice, Bob, and Charles.
        Each of the five voters has a first, second, and third choice,
        as listed in the following table:       
           \def\A{Alice}
           \def\B{Bob}
           \def\C{Charles}
        \begin{center}
          \begin{tabular}{l||l|l|l|l|l}
            Voter: & Voter 1 & Voter 2 & Voter 3 & Voter 4 & Voter 5 \\ \hline\hline
            1st choice: & \A & \B & \B & \C & \B \\
            2nd choice: & \C & \A & \C & \A & \A \\
            3rd choice: & \B & \C & \A & \B & \C \\
          \end{tabular}
        \end{center}
        Depending on who decides to run, the ballot the voters see in the booth will look different.
        For each of the following ballots, how many votes will each candidate receive?  Who will win?
        
        \begin{center}
          \begin{tabular}{p{2.5in}p{3in}}
                  (a)
                  \begin{tabular}{|cl|}\hline
                    \multicolumn{2}{|c|}{Who should be} \\
                    \multicolumn{2}{|c|}{president?} \\ \hline
                    $\square$ & Alice  \\
                    $\square$ & Bob     \\
                    $\square$ & Charles \\
                    \hline
                  \end{tabular}
                  \ifsolns
                    \soln Bob
                  \fi
            &
                  (c)
                  \begin{tabular}{|cl|}\hline
                    \multicolumn{2}{|c|}{Who should be} \\
                    \multicolumn{2}{|c|}{president?} \\ \hline
                    $\square$ & Alice  \\
                    $\square$ & Charles \\
                    \hline
                  \end{tabular}
                  \ifsolns
                    \soln Alice
                  \fi
            \\
          \phantom{X} & \\
          (b)     \begin{tabular}{|cl|}\hline
                    \multicolumn{2}{|c|}{Who should be} \\
                    \multicolumn{2}{|c|}{president?} \\ \hline
                    $\square$ & Alice  \\
                    $\square$ & Bob     \\
                    \hline
                  \end{tabular}
                  \ifsolns
                    \soln Alice
                  \fi
            &
          (d)     \begin{tabular}{|cl|}\hline
                    \multicolumn{2}{|c|}{Who should be} \\
                    \multicolumn{2}{|c|}{president?} \\ \hline
                    $\square$ & Bob     \\
                    $\square$ & Charles \\
                    \hline
                  \end{tabular}
                  \ifsolns
                    \soln Bob
                  \fi
          \end{tabular}
        \end{center}


%\clearpage
%%%%%%%%%%%
%%%%%%%%%%%Insert instructions on how to make a preference table here!
%%%%%%%%%%%
\clearpage
   %good for Borda count
   \def\gardenmatrix{%
	\large
        \begin{center}
          \begin{tabular}{|c|c|c|c|c|c|c|c} \hline
                       & \multicolumn{5}{c|}{Number of Ballots} \\
            Ranking    &  &  &  &  &      \\ \hline\hline
            1st choice & D & D & E & F & G     \\
            2nd choice & G & G & G & G & E     \\
            3rd choice & E & F & F & D & F     \\
            4th choice & F & E & D & E & D     \\ \hline
          \end{tabular}
        \end{center}}



  \item The Shell Rock Gardening Club is electing a new Master of the Greensward.
        Four candidates are running for the position: Danielle, Edgar, Frederika, and Gustav.
				
				    \item Here is the ballot:
                \begin{center}
                  \begin{tabular}{|cl|}\hline
                    \multicolumn{2}{|c|}{Who should be Master} \\
                    \multicolumn{2}{|c|}{of the Greensward?} \\ \hline
                    $\square$ & Danielle  \\
                    $\square$ & Edgar     \\
                    $\square$ & Frederika \\
                    $\square$ & Gustav    \\ \hline
                  \end{tabular}
                \end{center}
      
				
				The members cast the following ballots:
				\begin{center}
					
				\begin{tabular}{lllll}
				            Ranking     &             1st choice  &             2nd choice  &             3rd choice  &             4th choice \\\hline
Voter 1 & Danielle  & Gustav & Edgar  &  Frederika \\\hline
Voter 2 & Danielle  & Gustav &  Frederika  & Edgar \\\hline
Voter 3 & Edgar  & Gustav &  Frederika  & Danielle \\\hline
Voter 4 &  Frederika  & Gustav & Danielle  & Edgar \\\hline
Voter 5 & Gustav & Edgar      &  Frederika      & Danielle                   \\\hline
Voter 6 & Danielle  & Gustav & Edgar  &  Frederika \\\hline
Voter 7 & Danielle  & Gustav &  Frederika  & Edgar \\\hline
Voter 8 & Edgar  & Gustav &  Frederika  & Danielle \\\hline
Voter 9 &  Frederika  & Gustav & Danielle  & Edgar \\\hline
Voter 10 & Gustav & Edgar      &  Frederika      & Danielle                   \\\hline
Voter 11 & Danielle  & Gustav & Edgar  &  Frederika \\\hline
Voter 12 & Edgar  & Gustav &  Frederika  & Danielle \\\hline
Voter 13 &  Frederika  & Gustav & Danielle  & Edgar \\\hline
Voter 14 & Edgar  & Gustav &  Frederika  & Danielle\\\hline
				\end{tabular}
				
				\end{center}
        \begin{enumerate}
				\item 				Create a preference table for this election by counting the number of voters who have the same preference and putting that number above the appropriate ballot in the table.  You may use tally marks if you prefer.
				\gardenmatrix
				\ifsolns
				  \begin{tabular}{|c|c|c|c|c|c|c|c} \hline
                       & \multicolumn{5}{c|}{Number of Ballots} \\
            Ranking    & 3 &2  &4  &3  &2      \\ \hline\hline
        \end{tabular}
				\fi
\clearpage
\item    Who will be elected Master?  What percent of the vote will he or she have? \label{Ch1Prob3b}
							
                \ifsolns
                  \par\soln D, with $\frac{5}{14}\approx 35.7\%$
                \fi
								
              	\fillwithlines{\stretch{1}}
          \item As you look at the voters' preferences,
                who seems like the best choice to make the most voters happy? \label{Ch1Prob3c}
             
								\ifsolns
                  \par\soln G
                \fi
                \fillwithlines{\stretch{1}}
          \item If your answers to parts \ref{Ch1Prob3b} and \ref{Ch1Prob3c} are different,
                can you explain why?
                \fillwithlines{\stretch{2}}
\item    Write down how you would tell someone else how to create a preference table.
												
              	\fillwithlines{\stretch{1}}

          \item Can you think of a different voting system that would help the voters elect 
                your choice from part (b)?
								\fillwithlines{\stretch{1}}
        \end{enumerate}

\clearpage
  \item \textbf{Plurality voting: } \ifsolns When the candidate with the most first place votes wins. \fi 	\fillwithlines{\stretch{3}} \index{Plurality}
  \item \textbf{A \defnstyle{majority} is\ldots} \ifsolns more than 50\% of the votes. \fi 	\fillwithlines{\stretch{1}}
  \item \textbf{Borda count:} \ifsolns If you have $n$ candidates in the election, you give $n$ points for every first place votes, $n-1$ points for every second place vote, etc., and 1 point for every last place vote.  Add up the points for each candidate and the winner is the candidate with the most points. \fi 	\fillwithlines{\stretch{2}} \index{Borda count}

%%BORDA COUNT
\clearpage
  \item If the Shell Rock Gardening Club runs their election as a Borda count,
        who will win?
								\ifsolns
				  \begin{tabular}{|c|c|c|c|c|c|c|c} \hline
                       & \multicolumn{5}{c|}{Number of Ballots} \\
            Ranking    & 3 &2  &4  &3  &2      \\ \hline\hline
        \end{tabular}
				\fi

        \gardenmatrix
        \ifsolns
          \par\soln $\begin{array}{rlcr}
                       D: & 5\cdot 4+3\cdot 2+6\cdot 1 &=&32 \\
                       E: & 4\cdot 4 + 2\cdot 3+3\cdot 2+5\cdot 1 &=& 33 \\
                       F: & 3\cdot 4 + 8\cdot 2+3\cdot 1 &=& 31\\
                       G: & 2\cdot 4 + 12\cdot 3       &=&44 \\
                     \end{array}$
        \fi
				\vfill\vfill
	\fillwithlines{\stretch{1}}
%%RUNOFF ELECTIONS, LEADING INTO PLURALITY WITH ELIMINATION
%<*HWHEADER>
\HOMEWORK
%</HWHEADER>

%<*HOMEWORK>

Show all your work. 

\begin{Venumerate}
	\item Consider the following election ballots:
	\begin{multicols}{2}
		\begin{center}
			\begin{tabular}{llll}
			Voter 1 & A & B & C\\ \hline 
			Voter 2 & C & B & A\\ \hline 
			Voter 3 & B & C & A\\ \hline 
			Voter 4 & C & B & A\\ \hline 
			Voter 5 & C & B & A\\ \hline 
			Voter 6 & B & C & A\\ \hline 
			Voter 7 & A & B & C\\ \hline 
			Voter 8 & A & B & C\\ \hline 
			Voter 9 & A & B & C\\ \hline 
			Voter 10 & B & C & A\\ \hline 
			Voter 11 & C & B & A\\ \hline 
			Voter 12 & A & B & C\\ \hline 
			\end{tabular}
		\end{center}
		\begin{enumerate}
			\item Construct a preference table for the above ballots. 
			
			\begin{center} \large
          \begin{tabular}{|c|c|c|c|} \hline
                       & \multicolumn{3}{c|}{\# Ballots} \\
            Ranking     &  \makebox[.6cm]{} & \makebox[.6cm]{}  & \makebox[.6cm]{}       \\ \hline\hline  
            1st choice  & A &  &   \\ 
            2nd choice  & B &  &   \\ 
            3rd choice  & C &  &   \\ 
            \hline
          \end{tabular}
        \end{center}
\vfill

			\item How many votes would be required to win a \emph{majority} in this election?
         \vspace{1in}
			 Find the winner of an election held using each of the following voting schemes.
        (If it is a tie, say who tied.)
        \begin{enumerate}
          \item Using plurality vote.
                \solution*{ \fbox{A.}}\vspace{1in}
          \item Using Borda Count.
                \solution*{ \fbox{B.}}\vspace{2in}
						\end{enumerate}

		\end{enumerate}
			\end{multicols}
\hwnewpage
  \item Consider the following preference matrix: 
        \def\A{A} \def\B{B} \def\C{C} \def\D{D}
        \begin{center}
          \begin{tabular}{|c|c|c|c|c|c|c|c} \hline
                       & \multicolumn{5}{c|}{Number of Ballots} \\
            Ranking     & 8 & 6 & 8 & 5 & 1     \\ \hline\hline  
            1st choice  & \B & \C & \D & \B & \A \\ 
            2nd choice  & \D & \A & \C & \C & \B \\ 
            3rd choice  & \C & \D & \A & \A & \C \\ 
            4th choice  & \A & \B & \B & \D & \D \\ 
            \hline
          \end{tabular}
        \end{center}
        Find the winner of an election held using each of the following voting schemes.
        (If it is a tie, say who tied.)
        \begin{enumerate}
          \item Using plurality vote.
                \solution*{ \fbox{B.}}\vfill
          \item Using Borda Count.
                \solution*{ \fbox{C.}}\vfill
        \end{enumerate}

  \hwnewpage
  \item Consider the following preference matrix: 
        \def\A{A} \def\B{B} \def\C{C} \def\D{D}
        \begin{center}
          \begin{tabular}{|c|c|c|c|c|c|c|c} \hline
                       & \multicolumn{6}{c|}{Number of Ballots} \\
            Ranking     & 3 & 7 & 4 & 1 & 7 & 8     \\ \hline\hline  
            1st choice  & \B & \B & \C & \A & \C & \A \\ 
            2nd choice  & \C & \A & \B & \C & \A & \B \\ 
            3rd choice  & \A & \C & \A & \B & \B & \C \\ 
            \hline
          \end{tabular}
        \end{center}
        Find the winner of an election held using each of the following voting schemes.
        (If it is a tie, say who tied.)
        \begin{enumerate}
          \item Using plurality vote.
                \solution*{\fbox{C.}}\vfill
          \item Using Borda Count.
                \solution*{\fbox{Tied between A and B.}}\vfill
        \end{enumerate}
				
	\hwnewpage
  \item Consider the following preference matrix: 
        \def\A{A} \def\B{B} \def\C{C} \def\D{D} \def\E{E}
        \begin{center}
          \begin{tabular}{|c|c|c|c|c|c|c|c} \hline
                       & \multicolumn{3}{c|}{\# of Ballots} \\
            Ranking     & 3 & 5 & 4     \\ \hline\hline  
            1st choice  & \C & \B & \D \\ 
            2nd choice  & \E & \E & \C \\ 
            3rd choice  & \A & \A & \E \\ 
            4th choice  & \D & \C & \A \\ 
            5th choice  & \B & \D & \B \\ 
            \hline
          \end{tabular}
        \end{center}
        Find the winner of an election held using each of the following voting schemes.
        (If it is a tie, say who tied.)
        \begin{enumerate}
          \item Using plurality vote.
                \ifsolns  \soln \fbox{B.} \fi\vfill
          \item Using Borda Count.
                \ifsolns  \soln \fbox{E.} \fi\vfill
        \end{enumerate}
	\end{Venumerate}
\ENDHOMEWORK
  \item The Ashwaubenon Classical League is voting for its new proconsul,
        and the candidates are
          Publius,
          Quintus,
          Rufus,
          and Sextus.
                  \begin{center}
                    \begin{tabular}{|c|c|c|c|c|c|c|c} \hline
                                 & \multicolumn{4}{c|}{\# of Ballots} \\
                      Ranking    & 3 &10 & 9 & 8       \\ \hline\hline
                      1st choice & S & P & Q & R        \\
                      2nd choice & R & R & S & Q        \\
                      3rd choice & P & S & R & S        \\
                      4th choice & Q & Q & P & P        \\ \hline
                    \end{tabular}
                  \end{center}
        \begin{enumerate}
          \item Each League member votes for his favorite candidate.  
                What percentage of votes did the first-place candidate receive?
                \ifsolns
                  \par\soln P won with $33\%$ of the vote.
                \fi
                	\fillwithlines{\stretch{1}}
          \item Rather than elect a candidate who didn't receive a majority vote,
                the League holds a run-off election between the top two candidates.
                Who will win the run-off election?
                \ifsolns
                  \par\soln Q wins with 17 votes over P's 13 votes.
                \fi
                	\fillwithlines{\stretch{1}}
          \item Was anyone eliminated from the run-off who perhaps shouldn't have been?
                \ifsolns
                  \par\soln R was neck-and-neck with P and Q.
                \fi
                	\fillwithlines{\stretch{1}}
          \item Can you think of a better runoff system for the League than a runoff between just the top two candidates?
                What result does it give?
                \ifsolns
                  \par\soln Eliminate the worst each time.
                  $$
                  \begin{array}{rc} P & 10 \\ Q & 9 \\ R & 8 \\ S & 3\end{array}
                  \leadsto
                  \begin{array}{rc} P & 10 \\ Q & 9 \\ R & 11\end{array}
                  \leadsto
                  \begin{array}{rc} P & 10 \\ R & 20 \end{array}
                  \leadsto R
                  $$
                \fi
									\fillwithlines{\stretch{2}}
        \end{enumerate}

\clearpage
  \item \textbf{Plurality with elimination:} \ifsolns If no candidate has a majority of first place votes, eliminate the candidate, or candidates with the fewest first place votes.  Repeat until a candidate has a majority of first place votes. \fi \vfill\vfill	\fillwithlines{\stretch{1}}\index{Plurality with elimination}

%%PLURALITY WITH ELIMINATION
\clearpage
  \item The Federal Engineering Board is trying to decide on a flood prevention project
        for the metro area of Oak Rapids.  The options are
        \begin{itemize}\setlength{\itemsep}{0pt}\setlength{\parskip}{0pt}
          \item[(D)] A diversion through Marionville
          \item[(L)] A lock-and-dam system along the Oak River
          \item[(F)] Building floodwalls to $50'$ throughout the city
          \item[(C)] Relocating the community to higher ground in, say, Colorado
          \item[(G)] Annual goat sacrifices to Poseidon, God of Waters
        \end{itemize}
        The eighteen members of the Board have the following preferences:
        \begin{center}
          \begin{tabular}{|c|c|c|c|c|c|c|c} \hline
                       & \multicolumn{6}{c|}{Number of Ballots} \\
            Ranking    & 5 & 4 & 2 & 3 & 1 & 3 \\ \hline\hline
            1st choice & F & G & L & D & C & L \\
            2nd choice & D & D & F & G & D & F \\
            3rd choice & L & L & C & L & F & D \\
            4th choice & G & C & D & F & G & G \\
            5th choice & C & F & G & C & L & C \\ \hline
          \end{tabular}  
        \end{center}
        The Board's voting rules employ \emph{plurality with elimination}.
        \begin{enumerate}
          \item What will the vote tallies be after the first round of voting?
                Is there a winner yet?  If not, who should be eliminated?
                \ifsolns
                  \par\soln $\begin{array}{rr}D&3\\L&5\\F&5\\C&1\\G&4\end{array}$, eliminate C.
                \else
                  \fillwithlines{\stretch{1}}
                \fi
								
          \item What will the vote tallies be after the second round of voting?
                Is there a winner yet?  If not, who should be eliminated?
                \ifsolns
                  \par\soln $\begin{array}{rr}D&4\\L&5\\F&5\\G&4\end{array}$, eliminate D and G.
                \else
                  \fillwithlines{\stretch{1}}
                \fi
          \item What will the vote tallies be after the third round of voting?
                Is there a winner yet?  If not, who should be eliminated?
                \ifsolns
                  \par\soln $\begin{array}{rr}L&12\\F&6\end{array}$, eliminate F, L wins.
                \else
                  \fillwithlines{\stretch{1}}
                \fi
						
        \end{enumerate}
  \item If there are $n$ candidates in an election conducted by plurality with elimination, what is the most rounds of voting that could be required?
        \ifsolns
          \par\soln $n-1$ or $n-2$
				\else
				 	\fillwithlines{\stretch{1}}
        \fi
\vfill
%%PAIRWISE COMPARISON METHOD
\clearpage
  \item The Wartburg College Science Fiction Club is voting on who is the Best Starship Captain Ever.
        The candidates are:
           \def\AA{Solo}
           \def\BB{Kirk}
           \def\CC{Reynolds}
           \def\DD{Beeblebrox}
           \def\A{S}
           \def\B{K}
           \def\C{R}
           \def\D{B}
        \begin{itemize}
          \item Han Solo, \emph{Millennium Falcon}                     
          \item James T. Kirk, \emph{USS Enterprise}                   
          \item Malcolm Reynolds, \emph{Serenity}                      
          %\item Jean-Luc Picard, \emph{USS Enterprise}                
          \item Zaphod Beeblebrox, \emph{Heart of Gold}                
          %\item Phaethon Prime Rhadamanth, \emph{The Phoenix Exultant}
        \end{itemize}
        The preference table is the following: \index{preference table}
        \begin{center}
          \begin{tabular}{|c|c|c|c|c|c|c} \hline
                       & \multicolumn{5}{c|}{Number of Ballots} \\
            Ranking    &  4 &  2 &  3 &  2 &  3     \\ \hline\hline
            1st choice & \C & \D & \B & \A & \A     \\
            2nd choice & \A & \A & \C & \B & \C     \\
            3rd choice & \B & \B & \D & \C & \D     \\
            4th choice & \D & \C & \A & \D & \B     \\ \hline
          \end{tabular}
        \end{center}
        \begin{enumerate}
          \item Who would win in a head-to-head race between \AA\ and \BB?
                \ifsolns\par\soln
                  \begin{tabular}{lr}\AA & 11\\\BB & 3\end{tabular} 
                \else
									\fillwithlines{\stretch{1}}
								\fi
          \item Who would win in a head-to-head race between \AA\ and \CC?
                \ifsolns\par\soln
                  \begin{tabular}{lr}\AA & 7\\\CC & 7\end{tabular}
                \else
                  \fillwithlines{\stretch{1}}
                \fi
          \item Fill in the following table of who would win head-to-head races;
                if there are any ties, put both names in the box.
          
                %\def\myblank{\raisebox{0pt}[4ex][2ex]{\rule{1in}{0pt}}}
                \def\myblank#1{\raisebox{0pt}[4ex][2ex]{\makebox[1in][c]{\ifsolns #1 \fi}}}
                \begin{tabular}{c||c|c|c|ccccccc}
                         & \DD  & \CC  & \BB  \\ \hline \hline
                  {\AA}  & \myblank S    & \myblank{S-R}    & \myblank S   \\ \hline
                  {\BB}  & \myblank K    & \myblank{K-R}    \\ \cline{1-3}
                  {\CC}  & \myblank R   \\ \cline{1-2}
                \end{tabular}
          \item Who won the most head-to-head races? \\
                (Count any ties as $\frac{1}{2}$ a win for each contender.)
                \ifsolns
                  \par\soln
                  \begin{tabular}{rl}
                    S & 2.5 \\
                    K & 1.5 \\
                    R & 2 \\
                    B & 0
                  \end{tabular}
                \else
									\fillwithlines{\stretch{1}}
								\fi
        \end{enumerate}

\clearpage
  \item \textbf{Pairwise Comparison:} \ifsolns Place every candidate head to head against every other candidate.  Give one point for wins and 1/2 point for ties.  Candidate with most points wins.\else \vfill\fillwithlines{\stretch{1}} \fi \vfill \index{pairwise comparison}
  \item A tip for setting up the head-to-head races: \ifsolns Put the first $n-1$ candidates vertically.  Then, starting with the missing candidate and working backwards list $n-1$ candidates horizontally. \else \vfill\fillwithlines{\stretch{1}}\fi
\clearpage
  \item Our MA 106 classes are voting on where to take a math field trip.
        The options are:
        \begin{itemize}
          %\item[(N)] The National Security Agency, largest employer of mathematicians in the world
          \item[(E)] The mudflats of Egypt, birthplace of geometry
          %\item[(L)] Las Vegas, a city that runs on statistics
          %\item[(F)] Florence, home of Renaissance perspective
          \item[(A)] Athens, cradle of logical thought
          \item[(M)] Moose Lake Credit Union, dispenser of mortgages
          \item[(K)] K\"onigsberg, for a walking tour of the bridges
          \item[(P)] Papa John's Pizza, home of yummy garlic sauce
        \end{itemize}
        The preference table is the following.
        \begin{center}
          \begin{tabular}{|c|c|c|c|c|c|c|c} \hline
                       & \multicolumn{6}{c|}{Number of Ballots} \\
            Ranking    & 5 & 4 & 4 & 3 & 2 & 1 \\ \hline\hline
            1st choice & K & M & M & E & P & K \\
            2nd choice & E & K & E & P & K & A \\
            3rd choice & A & A & A & K & E & M \\
            4th choice & P & E & K & M & M & P \\
            5th choice & M & P & P & A & A & E \\ \hline
          \end{tabular}
        \end{center}
        We decide to use the pairwise comparison method to decide.
        Where will we go?
        \ifsolns
          \par\soln
          \begin{tabular}{c|cccc}
               & E & A & M & K  \\ \hline
             P & E & A & P & K  \\ 
             K & K & K & K &    \\ 
             M & E & M &   &    \\
             A & E &   &   &    \\
          \end{tabular}
          \qquad
          \begin{tabular}{cr}
            E & 3 \\
            A & 1 \\
            M & 1 \\
            K & 4 \\
            P & 1 \\
          \end{tabular}
          \quad
          K wins
        \else
          \fillwithlines{\stretch{1}}
        \fi
  \item If you use the pairwise comparison method with $n$ candidates, how many head-to-head races are there?
        \ifsolns
          \par\soln
          $\binom{n}{2} = \frac{n(n-1)}{2}$
        \fi\fillwithlines{\stretch{1}}

\end{enumerate}

%</WORKSHEETS>

%<*HWHEADER>
\HOMEWORK
%</HWHEADER>

%<*HOMEWORK>

Show all your work.  In particular, for plurality with elimination, show the results of each round of voting;
for pairwise comparison, show the result of each head-to-head vote.

\begin{Venumerate}
	\item Consider the following preference matrix:
        \def\A{A} \def\B{B} \def\C{C} \def\D{D} \def\E{E}
        \begin{center} 
          \begin{tabular}{|c|p{1cm}|p{1cm}|p{1cm}|} \hline
                       & \multicolumn{3}{c|}{Number of Ballots} \\
            Ranking     &  5 &  4 &  3      \\ \hline\hline  
            1st choice  & \A & \C & \B  \\ 
            2nd choice  & \B & \B & \C  \\ 
            3rd choice  & \C & \A & \A  \\ 
            \hline
          \end{tabular}
        \end{center}

		\begin{enumerate}
			%\item Construct a preference matrix for the above ballots. \vfill
			\item How many votes would be required to win a \emph{majority} in this election?
         \vfill
			 Find the winner of an election held using each of the following voting schemes.
        (If it is a tie, say who tied.)
        \begin{enumerate}
      
          \item Find the winner using plurality with elimination.
                \ifsolns
                  \soln \fbox{C.}
                \fi\vfill
          \item Find the winner using pairwise comparison.
                \ifsolns
                  \soln \fbox{B.}
                \fi\vfill
						\end{enumerate}
		\end{enumerate}
		\hwnewpage
		
  \item Consider the following preference matrix:
        \def\A{A} \def\B{B} \def\C{C} \def\D{D} \def\E{E}
        \begin{center} 
          \begin{tabular}{|c|c|c|c|c|c|c|c} \hline
                       & \multicolumn{6}{c|}{Number of Ballots} \\
            Ranking     &  3 &  4 &  4 &  9 &  3 &  6     \\ \hline\hline  
            1st choice  & \D & \D & \A & \E & \B & \A \\ 
            2nd choice  & \C & \C & \B & \C & \E & \B \\ 
            3rd choice  & \E & \E & \E & \B & \D & \C \\ 
            4th choice  & \B & \A & \D & \A & \A & \D \\ 
            5th choice  & \A & \B & \C & \D & \C & \E \\ 
            \hline
          \end{tabular}
        \end{center}
        This race is being decided by plurality vote.
        \begin{enumerate}
          \item How many voters are voting in this election? 
                \solution*{$3+4+4+9+3+6=\boxed{29.}$}\vfill
          \item How many votes would be required to win a \emph{majority} in this election?
                \solution*{$29/2 = 14.5$, so it takes \fbox{15 votes.}}\vfill
          \item Who will win this election?  How many votes will that person get?  Does that candidate have a majority?
                \esolution*{Candidate \A\ wins with 10 votes, but that is not a majority.}\vfill
          \item Suppose candidate \B\ drops out of the race.  Who would win the election?
                \esolution*{Now candidate \E\ wins (with 12 votes).}\vfill
          \item Suppose everyone drops out of the race except \A\ and \C.  Who will win the election?
                \esolution*{\A\ gets 13 votes and \C\ gets 16 votes, so \C\ wins.}\vfill
        \end{enumerate}

\hwnewpage
  \item Consider the following preference matrix: 
        \def\A{A} \def\B{B} \def\C{C} \def\D{D}
        \begin{center}
          \begin{tabular}{|c|c|c|c|c|c|c|c} \hline
                       & \multicolumn{5}{c|}{Number of Ballots} \\
            Ranking     & 8 & 6 & 8 & 5 & 1     \\ \hline\hline  
            1st choice  & \B & \C & \D & \B & \A \\ 
            2nd choice  & \D & \A & \C & \C & \B \\ 
            3rd choice  & \C & \D & \A & \A & \C \\ 
            4th choice  & \A & \B & \B & \D & \D \\ 
            \hline
          \end{tabular}
        \end{center}
       \item Find the winner of an election held using each of the following voting schemes.
        (If it is a tie, say who tied.)
        \begin{enumerate}
          \item Find the winner using plurality with elimination.
                \ifsolns
                  \soln \fbox{\parbox{.5\textwidth}{Tied between B and D.}} \newline
                \fi\vfill
          \item Find the winner using pairwise comparison.
                \ifsolns
                  \soln \fbox{D.}
                \fi\vfill
        \end{enumerate}

  \hwnewpage
  \item Consider the following preference matrix: 
        \def\A{A} \def\B{B} \def\C{C} \def\D{D}
        \begin{center}
          \begin{tabular}{|c|c|c|c|c|c|c|c} \hline
                       & \multicolumn{6}{c|}{Number of Ballots} \\
            Ranking     & 3 & 7 & 4 & 1 & 7 & 8     \\ \hline\hline  
            1st choice  & \B & \B & \C & \A & \C & \A \\ 
            2nd choice  & \C & \A & \B & \C & \A & \B \\ 
            3rd choice  & \A & \C & \A & \B & \B & \C \\ 
            \hline
          \end{tabular}
        \end{center}
        Find the winner of an election held using each of the following voting schemes.
        (If it is a tie, say who tied.)
        \begin{enumerate}
          \item Find the winner using plurality with elimination.
                \solution*{\fbox{B.}}\vfill
          \item Find the winner using pairwise comparison.
                \solution*{\fbox{A.}}\vfill
        \end{enumerate}
				
	\hwnewpage
  \item Consider the following preference matrix: 
        \def\A{A} \def\B{B} \def\C{C} \def\D{D} \def\E{E}
        \begin{center}
          \begin{tabular}{|c|c|c|c|c|c|c|c} \hline
                       & \multicolumn{3}{c|}{\# of Ballots} \\
            Ranking     & 3 & 5 & 4     \\ \hline\hline  
            1st choice  & \C & \B & \D \\ 
            2nd choice  & \E & \E & \C \\ 
            3rd choice  & \A & \A & \E \\ 
            4th choice  & \D & \C & \A \\ 
            5th choice  & \B & \D & \B \\ 
            \hline
          \end{tabular}
        \end{center}
        Find the winner of an election held using each of the following voting schemes.
        (If it is a tie, say who tied.)
        \begin{enumerate}
          \item Find the winner using plurality with elimination.
                \ifsolns  \soln \fbox{D.} \fi\vfill
          \item Find the winner using pairwise comparison.
                \ifsolns  \soln \fbox{C.} \fi\vfill
        \end{enumerate}
\hwnewpage
  \item The AP college football poll is a ranking of the top 25 college football teams in the country and is one of the key polls used for the BCS\footnote{Bowl Championship Series}. The voters in the AP poll are a group of sportswriters and broadcasters chosen from across the country. The top 25 teams are ranked using a Borda count: each first-place vote is worth 25 points, each second-place vote is worth 24 points, each third-place vote is worth 23 points, and so on. The following table shows the ranking and total points for each of the top three teams at the end of the 2006 regular season. (The remaining 22 teams are not shown here because they are irrelevant to this exercise.)
  \begin{center}
	\begin{tabular}{ll}
	\hline
Team & Points \\\hline
1. Ohio State & 1625 \\\hline
2. Florida & 1529 \\\hline
3. Michigan & 1526 \\\hline
\end{tabular}
\end{center}
  \begin{enumerate}
	\item  Given that Ohio State was the unanimous first-place choice of all the voters, find the number of voters that participated in the poll.
	\solution*{\fbox{65}}\vfill
	
	\item Find the number of second and third-place votes for Florida.
		\solution*{Hint: Since Ohio State was the unanimous first-place choice and we are told the other 22 teams are irrelevant, then we can assume that all the second and third place votes went either to Ohio State or Michigan.  This also means that the voters who didn't put Ohio State second placed them third.  This is all the information you need to solve the problem.}
		\ifsolns
			\fbox{34, 31}
		\fi \vfill
	\item Find the number of second and third-place votes for Michigan.
		\ifsolns
			\fbox{31, 34}
		\fi\vfill
\end{enumerate}
\end{Venumerate}


\ENDHOMEWORK %\ENDHOMEWORK %</HOMEWORK>
%<*WORKSHEETS>


\section{Voting Paradoxes and Problems}

\begin{enumerate}
  \item Of the four voting systems we've studied, which is best?  Why?
        \fillwithlines{\stretch{1}}

  %PLURALITY METHOD FAILS HEAD-TO-HEAD CRITERION
  \item The Des Moines Running Club is electing a new team captain.
        There are three candidates:  Irving, Joseph, and Karl.
        The preference table is the following.
        \begin{center}
          \begin{tabular}{|c|c|c|c|c|c|c|c} \hline
                       & \multicolumn{3}{c|}{\# of Ballots} \\
            Ranking    & 4 & 3 & 2 \\ \hline\hline
            1st choice & J & I & K  \\
            2nd choice & I & J & I  \\
            3rd choice & K & K & J  \\ \hline
          \end{tabular}
        \end{center}
        \begin{enumerate}
          \item If the club uses the plurality method,
                who will win the election? \ifsolns\fbox J \fi
                \fillwithlines{\stretch{1}}
          \item If it were a head-to-head race between Irving and Joseph, who would win? \ifsolns\fbox I \fi
                \fillwithlines{\stretch{1}}
          \item If it were a head-to-head race between Irving and Karl, who would win? \ifsolns\fbox I \fi
                \fillwithlines{\stretch{1}}
          \item If it were a head-to-head race between Joseph and Karl, who would win? \ifsolns\fbox J \fi
                \fillwithlines{\stretch{1}}
          \item Why does the plurality method's result seem unfair in this election?
                \ifsolns Irving defeats Joseph and Karl in head-to-head matches, but Joseph wins via plurality.\fi
                \fillwithlines{\stretch{1}}
        \end{enumerate}


\clearpage

  \item \boxedblank{\textbf{Head-to-Head Criterion:} \ifsolns \par\soln A candidate who beats all the other candidates head to head should win the election. \fi          \fillwithlines{\stretch{1}}} \index{criterion!head to head}

  %BORDA COUNT FAILS MAJORITY CRITERION
  \item The school board is voting whether to award the contract for building a new elementary school
        to Aardvark Architects,
        to Bob's Builders,
        to the Confident Construction Company,
        or to Davenfield Developers.
        \begin{center}
          \begin{tabular}{|c|c|c|c|c|c|c|c} \hline
                       & \multicolumn{4}{c|}{\# of Ballots} \\
            Ranking    & 5 & 3 & 3 & 1   \\ \hline\hline
            1st choice & B & B & C & A  \\
            2nd choice & C & C & A & D   \\
            3rd choice & D & A & D & C   \\
            4th choice & A & D & B & B   \\ \hline
          \end{tabular}
        \end{center}
      \begin{enumerate}
        \item The school board votes using a Borda count.  \\
              Which company will get the contract? %Company C
              \ifsolns
                \par\soln $\begin{array}{clcr}
                             A & 4+9+6+5 &=& 24 \\
                             B & 32+4    &=& 36 \\
                             C & 24+12+2 &=& 38 \\
                             D & 3+16+3  &=& 25 \\
                           \end{array}$ 
                \qquad so C wins
              \fi
              \fillwithlines{\stretch{2}}
        \item Does this seem fair?  Why or why not?
              \ifsolns
                \par\soln No; two-thirds of the voters picked B as their first choice!
              \else
                \fillwithlines{\stretch{1}}
              \fi
      \end{enumerate}  

\clearpage
  \item \boxedblank{\textbf{Majority Criterion:} \ifsolns A candidate that wins a majority of first place votes should win the election. \fi} \index{criterion!majority}
  %PLURALITY WITH ELIMINATION FAILS MONOTONICITY

\ifodd\value{page}\clearpage\fi

  \item The board of regents of
        Western Iowa State at Sema
        will be voting tomorrow for the University's new president
        from four candidates: 
           Zehnpfennig,     \def\A{Z}\def\AA{Zehnpfennig}%
           Gersbach,        \def\B{G}\def\BB{Gersbach}%
           Haider,          \def\C{H}\def\CC{Haider}%
           and Rentmeester. \def\D{R}\def\DD{Rentmeester}%
        As of this moment, the voters' preferences are as follows:
        \begin{center}
          \begin{tabular}{|c|c|c|c|c|c|c|c} \hline
                       & \multicolumn{4}{c|}{\# of Ballots} \\
            Ranking    & 7 & 5 & 4 & 1   \\ \hline\hline
            1st choice & \A & \C & \B & \D  \\
            2nd choice & \D & \A & \C & \B   \\
            3rd choice & \B & \B & \D & \A   \\
            4th choice & \C & \D & \A & \C   \\ \hline
          \end{tabular}
        \end{center}
        The WIS regents use the plurality with elimination method.
        \begin{enumerate}
          \item If the vote were held today, who would be chosen as university president? 
                \ifsolns
                  \par\soln
                  \[\begin{array}{cc}Z& 7 \\ R&1 \\ G&4 \\ H&5\end{array}
                    \leadsto
                    \begin{array}{cc}Z&7 \\ G&5 \\ H&5 \end{array}
                    \leadsto Z\]
                \else
								\vfill
                  \fillwithlines{\stretch{1}}
                \fi
        \clearpage
          \item After thinking it over in bed tonight, 
                the single voter in the last column decides 
                that \AA\ really is the best candidate after all, and so his preferences change to \A,\D,\B,\C.
                Thus when the election is held tomorrow, the preference table is
                        \newlength{\mywidth}
                        \def\no#1{\settowidth{\mywidth}{#1}%
                                  %\makebox[0pt][l]{\raisebox{1.0ex}{\rule{\mywidth}{0.5pt}}}%
                                  %\makebox[0pt][l]{\raisebox{0.8ex}{\rule{\mywidth}{0.5pt}}}%
                                  \makebox[0pt][l]{\makebox[\mywidth][c]{$\times$}}%
                                  %\makebox[0pt][l]{\raisebox{0.6ex}{\rule{\mywidth}{0.5pt}}}%
                                  #1}
                        \begin{center}
                          \begin{tabular}{|c|c|c|c|c|c|c|c} \hline
                                       & \multicolumn{4}{c|}{\# of Ballots} \\
                            Ranking    & 7 & 5 & 4 & 1   \\ \hline\hline
                            1st choice & \A & \C & \B & \no{\D}  \A \\
                            2nd choice & \D & \A & \C & \no{\B}  \D \\
                            3rd choice & \B & \B & \D & \no{\A}  \B \\
                            4th choice & \C & \D & \A & \no{\C}  \C \\ \hline
                          \end{tabular}
                        \end{center}
                Who will be elected president?
                \ifsolns
                  \par\soln
                  $$\begin{array}{cc}Z& 8 \\ R&0 \\ G&4 \\ H&5\end{array}
                    \leadsto
                    \begin{array}{cc}Z& 8 \\ G&4 \\ H&5\end{array}
                    \leadsto
                    \begin{array}{cc}Z&8 \\ H&9 \end{array}
                    \leadsto H$$
                \else
                  \fillwithlines{\stretch{2}}
                \fi
                
          \item What is odd about your answers to (a) and (b)?
                \fillwithlines{\stretch{1}}
        \end{enumerate}

\item \boxedblank{\textbf{Monotonicity Criterion:} \ifsolns If the only changes in voting are in favor of the winning candidate, that candidate should still win the election. \fi} \fillwithlines{\stretch{1}}\index{criterion!monotonicity}

\clearpage
%CONDORCET WINNER MAY NOT EXIST---IN FACT, NONTRANSITIVITY
  \item Voters are trying to decide what to do with a \$50 million surplus in the state budget.
        Their options are: 
        \begin{itemize}
          \item[(T)] Give the money back to taxpayers as property tax relief.
          \item[(R)] Spend it on road construction and other infrastructure.
          \item[(H)] Build a new state historical building.
        \end{itemize}
        Here is the preference table:
                        \begin{center}
                          \begin{tabular}{|c|c|c|c|c|c|c|c} \hline
                                       & \multicolumn{3}{c|}{Number of Ballots} \\
                            Ranking    &  70,000 & 50,000 & 40,000    \\ \hline\hline
                            1st choice & T & R & H \\
                            2nd choice & R & H & T \\
                            3rd choice & H & T & R \\ \hline
                          \end{tabular}
                        \end{center}
        \begin{enumerate}
          \item The election is run by pairwise comparison.  What is the result?
                \ifsolns
                  \par\soln $R>H>T>R$.
                \fi
                \vfill\fillwithlines{\stretch{2}}
          \item What is odd about your results from part (a)?
					\ifsolns Try having candidate $R$ drop out of the race. \fi\fillwithlines{\stretch{1}}
        \end{enumerate}
        
\clearpage

  \item \boxedblank{\textbf{Irrelevant Alternatives Criterion:} \ifsolns \par\soln If a non-winning candidate leaves the election, the winner should not change, \fi\fillwithlines{\stretch{1}}} \index{criterion!irrelevant alternatives}

  \item Approval Voting: \index{approval voting}


	\ifsolns Vote for all candidates you approve of. \fi
        \fillwithlines{\stretch{1}}

  \item \boxedblank[2in]{\textbf{Arrow's Impossibility Theorem:} \ifsolns There is no way of counting votes that does not violate at least one of the criteria except for a Dictator. \fi\fillwithlines{\stretch{1}}} \index{Arrow's Impossibility Theorem}
%%%%%%%%%%%%%%%%%%%%%%%%%%%%%%%%%%%%%%%%%%%%%%%%%%%%%%%%%%%%%%%%%%%%%%%%%%%%%%%%%%%%%%%%%%




\end{enumerate}


%</WORKSHEETS>

%<*HWHEADER>
\cleartooddpage
\HOMEWORK
%</HWHEADER>

%<*HOMEWORK>


\begin{Venumerate}
  \item Consider the following preference matrix:
        \def\A{A} \def\B{B} \def\C{C} \def\D{D}
        \begin{center}
          \begin{tabular}{|c|c|c|c|c|c|c|c} \hline
                       & \multicolumn{5}{c|}{Number of Ballots} \\
            Ranking     & 4 & 10 & 2 & 3 & 5     \\ \hline\hline  
            1st choice  & \C & \D & \A & \C & \A \\ 
            2nd choice  & \D & \C & \C & \B & \B \\ 
            3rd choice  & \B & \A & \B & \A & \D \\ 
            4th choice  & \A & \B & \D & \D & \C \\ 
            \hline
          \end{tabular}
        \end{center}
        \begin{enumerate}
          \item Does the Borda Count violate the Majority Criterion for this particular preference matrix?
                \solution*{
                  The Borda count tally is
                  \begin{tabular}[c]{|ll|}\hline
                    A & 58 \\
                    B & 46 \\
                    C & 69 \\
                    D & 67 \\ \hline
                  \end{tabular}, \\
                  while a plurality vote yields
                  \begin{tabular}[c]{|ll|}\hline
                    A & 7 \\
                    B & 0 \\
                    C & 7 \\
                    D & 10 \\ \hline
                    total & 24 \\ \hline
                  \end{tabular}
                  \par
                  Thus \fbox{no,} the majority criterion is not violated since no candidate had a majority of the 24 votes.
                  The majority criterion would only be violated if 
                  \emph{both} (a) some candidate had a majority, and (b) the Borda count elected someone different.
                } \vfill          \fillwithlines{\stretch{1}}
          \item Does the Borda Count violate the Head-to-Head Criterion for this particular preference matrix?
                \solution*{
                  \fbox{Yes.} Candidate D would beat each of A, B, and C in head-to-head contests,
                  but the Borda count elects someone else.
                }\vfill           \fillwithlines{\stretch{1}}
        \end{enumerate}

\hwnewpage
  \item Consider the following preference matrix:
        \def\A{A} \def\B{B} \def\C{C} \def\D{D}
        \begin{center}
          \begin{tabular}{|c|c|c|c|c|c|c|c} \hline
                       & \multicolumn{6}{c|}{Number of Ballots} \\
            Ranking     & 11 & 2 & 5 & 1 & 8 & 4     \\ \hline\hline  
            1st choice  & \C & \D & \C & \D & \A & \B \\ 
            2nd choice  & \A & \A & \A & \A & \D & \D \\ 
            3rd choice  & \D & \C & \B & \B & \C & \C \\ 
            4th choice  & \B & \B & \D & \C & \B & \A \\ 
            \hline
          \end{tabular}
        \end{center}
        \begin{enumerate}
          \item Does the Borda Count violate the Majority Criterion for this particular preference matrix?
                \ifsolns
                  \par\soln The Borda count tally is
                  \begin{tabular}[c]{|ll|}\hline
                    A & 93 \\
                    B & 49 \\
                    C & 93 \\
                    D & 75 \\ \hline
                  \end{tabular}
                  \\ while a plurality vote yields
                  \begin{tabular}[c]{|ll|}\hline
                    A & 8 \\
                    B & 4 \\
                    C & 16 \\
                    D & 3 \\ \hline
                    total & 31 \\ \hline
                  \end{tabular}
                  \par
                  Thus \fbox{yes,} the majority criterion is violated, since candidate C has a majority of the 31 votes,
                  but the Borda Count does not elect him; instead, the Borda count has C tying with A.
                \else \vfill          \fillwithlines{\stretch{1}} \fi
          \item Does the Borda Count violate the Head-to-Head Criterion for this particular preference matrix?
                \ifsolns
                  \par\soln
                  \fbox{Yes,} since C beats A, B, and D in head-to-head contests but the Borda Count does not elect him.
                \else \vfill           \fillwithlines{\stretch{1}}\fi
        \end{enumerate}

\hwnewpage
  \item Consider the following preference matrix:
        \def\A{A} \def\B{B} \def\C{C} \def\D{D}
        \begin{center}
          \begin{tabular}{|c|c|c|c|c|c|c|c} \hline
                       & \multicolumn{4}{c|}{\# of Ballots} \\
            Ranking     & 9 & 6 & 8 & 5      \\ \hline\hline  
            1st choice  & \B & \B & \D & \D \\ 
            2nd choice  & \D & \D & \B & \B \\ 
            3rd choice  & \C & \A & \A & \C \\ 
            4th choice  & \A & \C & \C & \A \\ 
            \hline
          \end{tabular}
        \end{center}
        \begin{enumerate}
          \item How many points will each candidate receive in a Borda count?  Who will win?
                \ifsolns
                  \par\soln
                  The point tallies are
                  \begin{tabular}[c]{|ll|}\hline
                    A & 42 \\
                    B & 99 \\
                    C & 42 \\
                    D & 97 \\ \hline
                  \end{tabular}
                  so \fbox{B wins.}
                \else \vfill          \fillwithlines{\stretch{1}}\fi
          \item The five voters in the last column really do think \D\ is the best candidate and \B\ is the second-best.
                However, they decide to be sneaky and lie on their Borda count ballots,
                claiming they think \B\ is the worst candidate;
                in other words, they say they prefer \D, \C, \A, and \B\ in that order.
                Now how many points will each candidate receive in a Borda count?  Who will win?
                \ifsolns
                  \par\soln
                  Now the point tallies become
                  \begin{tabular}[c]{|ll|}\hline
                    A & 47 \\
                    B & 89 \\
                    C & 47 \\
                    D & 97 \\ \hline
                  \end{tabular}
                  so \fbox{D wins.}
                \else \vfill           \fillwithlines{\stretch{1}}\fi
          \item Explain in a few complete sentences how these voters manipulated the Borda count and why it is unfair.
                \ifsolns
                  \par\soln These voters filled out their ballots deceptively, downgrading the candidate they opposed 
                  in order to make their own candidate do well by comparison.
                  This is unfair because it gives dishonest voters more power than honest ones.
                  (Graders, be generous with students' explanations!)
                \else \vfill          \fillwithlines{\stretch{1}}\fi
        \end{enumerate}
				
\hwnewpage
  %COST-CONSCIOUS VOTERS, courtesy of Lisa Moats's REU work
  \item Voters go to the polls to vote on three propositions simultaneously:
        \begin{description}
          \item[Proposition~1.] Spend \$200,000 to plant new flowers and trees at the zoo.
          \item[Proposition~2.] Spend \$400,000 to build a children's playground at the zoo.
          \item[Proposition~3.] Spend \$500,000 to add a panda exhibit to the zoo.
        \end{description}
        All the voters love the zoo, plants, children, and pandas.
        They're all fiscally responsible, and only differ in how much money they want to spend.
        \begin{enumerate}
          \item Alice is one of a thousand voters who want to improve the zoo,
                but don't want to spend more than \$1,000,000 total.
                She wants to spend as close that limit as possible, without going over.
                Which propositions will Alice and her bloc vote for?
                \solution*{Alice et al.~will vote for \fbox{Propositions 2 and 3.}}
          \item Zeke is one of a thousand voters who want to improve the zoo,
                but don't want to spend more than \$800,000 total.
                He wants to spend as close that limit as possible, without going over.
                Which propositions will Zeke and his bloc vote for?
                \solution*{Zeke et al.~will vote for \fbox{Propositions 1 and 3.}}
          \item Shadrach is one of a thousand voters who want to improve the zoo,
                but don't want to spend more than \$600,000 total.
                He wants to spend as close that limit as possible, without going over.
                Which propositions will Shadrach and his bloc vote for?
                \solution*{Shadrach et al.~will vote for \fbox{Propositions 1 and 2.}}
          \item Fill out the following table with the votes---either ``Yes'' or ``No.''\par
                \def\myblank{\raisebox{0pt}[4ex][2ex]{\rule{1in}{0pt}}}
                \ifsolns
                  \def\myblank#1{{\textcolor{red}#1}}
                \else
                  \def\myblank#1{\phantom{X}}
                \fi
                \begin{tabular}{l||c|c|c|}
                  Voters          & Alice et al. & Zeke et al. & Shadrach et al. \\
                  Number of votes & 1000         & 1000        & 1000 \\ \hline\hline
                  Proposition 1   & \myblank N   & \myblank Y  & \myblank Y \\ \hline
                  Proposition 2   & \myblank Y   & \myblank N  & \myblank Y \\ \hline
                  Proposition 3   & \myblank Y   & \myblank Y  & \myblank N \\ \hline
                \end{tabular}
          \item Which of the three propositions will pass?  (Only a majority vote is needed for each proposition.)
                \solution*{\fbox{All three} propositions will pass with 2000 of the 3000 votes.}          \fillwithlines{\stretch{1}}
          \item How many of the 3,000 voters are happy with the result?
                \solution*{Since this will cost taxpayers \$1,100,000, \fbox{none of the voters} will be happy.}          \fillwithlines{\stretch{1}}
          \item Explain in a few complete sentences what is odd or paradoxical about this situation.
                \solution*{Each of the voters deliberately voted to limit spending, 
                  yet they collectively ended up spending more than any of them wanted.
                  \ifgradersolns
                    (Graders, please be generous with their explanations.)
                  \fi
                }          \fillwithlines{\stretch{2}}
        \end{enumerate}
\end{Venumerate}


\ENDHOMEWORK %</HOMEWORK>

%<*WORKSHEETS>



\cleartooddpage

\section{Weighted Voting Systems}

\begin{enumerate}
  \item Notes on weighted voting systems: \index{weighted voting}
	
	\begin{itemize}
		\item $P_i$ \vfill
		\item $w_i$ \vfill
		\item $q$\vfill
		\item $[q|w_1,w_2,\dots,w_n]$ \vfill
	\end{itemize}

\ifsolns \par\soln
	Weighted voting systems are when there are $n$ voters labeled $P_1\dots P_n$, and each voter casts a vote with a certain weight, $w_i$.  Think of this as having each voter being able to cast $w_i$ votes all for the same decision.  In order for a motion to pass, the sum of all the votes needs to be at least $q$, the quota.  We need the quota to be more than 50\% of the sum of all the votes and no more than the sum of all the votes, but it can be anything in between.  We give a weighted voting system as 
	$[q|w_1,w_2,\dots,w_n]$.
\else          \vfill \fillwithlines{\stretch{1}}\fi
  \clearpage
  \item In a weighted voting system with weights $[30, 29, 16, 8, 3, 1]$,
        if a two-thirds majority of votes is needed to pass a motion, what is the quota? \index{quota}
        \ifsolns
					\fbox{58}
				\else
					          \fillwithlines{\stretch{1}}
				\fi

  \item Consider the weighted voting system $[14, 9, 8 ,5]$.
		\begin{enumerate}
          \item What is the largest reasonable quota for this system?
                  \ifsolns
					\fbox{36}
				\else
					          \fillwithlines{\stretch{1}}
				\fi
          \item What is the smallest reasonable quota for this system?
                  \ifsolns
					\fbox{19}
				\else
					          \fillwithlines{\stretch{1}}
				\fi
		\end{enumerate}
  \item Consider the weighted voting system $[20|7,5,4,4,2,2,2,1,1]$.
        \begin{enumerate}
          \item How many voters are there?
        \ifsolns
					\fbox{9}
				\else
					          \fillwithlines{\stretch{1}}
				\fi
          \item What is the quota?
        \ifsolns
					\fbox{20}
				\else
					          \fillwithlines{\stretch{1}}
				\fi
          \item What is the weight for voter $P_2$?
         \ifsolns
					\fbox{5}
				\else
					          \fillwithlines{\stretch{1}}
				\fi
         \item If the first 4 voters vote for a motion and the rest vote against, does the motion pass?
                \ifsolns
					\fbox{Yes}
				\else
					          \fillwithlines{\stretch{1}}
				\fi
        
          \item If $P_1$ and $P_2$ vote against a motion, will the motion pass?
        \ifsolns
					\fbox{No}
				\else
					          \fillwithlines{\stretch{1}}
				\fi
        \end{enumerate}

\clearpage

  \item What is peculiar about each of the following weighted voting systems?
	
        \begin{enumerate}
        \item $[20|10, 10, 9]$
					\ifsolns
						\fbox{Cannot win without $P_1$ and $P_2$.}
					\else
						\vfill          \fillwithlines{\stretch{1}}
					\fi
				%\end{enumerate}
		
        \item $[7|4, 2, 1]$
        \ifsolns
					\fbox{Everyone must vote for the motion to pass}
				\else
					\vfill          \fillwithlines{\stretch{1}}
				\fi
        %\end{enumerate}

		\item $[51|50, 49, 1]$
        \ifsolns
					\fbox{First voter must vote for the motion for the motion to pass.}
				\else
					\vfill          \fillwithlines{\stretch{1}}
				\fi
        %\end{enumerate}
		
        \item $[6|6, 2, 1, 1]$
        \ifsolns
					\fbox{First voter is the only one who makes a difference.}
				\else
					\vfill          \fillwithlines{\stretch{1}}
				\fi
        %\end{enumerate}
		
		\end{enumerate}
		
\pagebreak
    \item A \defnstyle{dummy} is \ldots 
		    \ifsolns
					\par\soln A voter whose vote does not make any difference on whether a motion passes or not.
				\fi
        %\end{enumerate}
          \fillwithlines{\stretch{1}} \index{dummy}
	
    \item A \defnstyle{dictator} is \ldots 
		    \ifsolns
					\par\soln A voter where their vote is the only reason a motion passes.
				\fi
		          \fillwithlines{\stretch{1}} \index{dictator}
    \item A voter has \defnstyle{veto power} if \ldots 
		    \ifsolns
					\par\soln A voter whose must vote for a motion in order for it to pass.
				\fi
		          \fillwithlines{\stretch{1}} \index{veto power}
    \item In the weighted voting system $[12|9,5,4,2]$, are there any dummies or dictators?
                    \vspace{1.5in}

    \item In designing a weighted voting system $[q|6,5,4,3,2,1]$, what is the largest quota $q$
          you could pick without giving veto power to anyone?
                    \vspace{1.5in}
    
    \item In the weighted voting system $[q|8,5,4,1]$, if every voter has veto power, what is the quota $q$?
                    \vspace{1.5in}

\clearpage
\item A committee has four members ($P_1, P_2, P_3,$ and $P_4$).  In this committee, $P_1$ has twice as many votes as $P_2$; $P_2$ has twice as many votes as $P_3$: $P_3$ has twice as many votes as $P_4$.  Describe the committee as a weighted voting system when the requirements to pass a motion are
\begin{enumerate}
	\item at least two-thirds of the votes 
		\ifsolns \fbox{[10|8,4,2,1]} \else \vfill           \fillwithlines{\stretch{1}} \fi
	\item more than two-thirds of the votes 
		\ifsolns \fbox{[11|8,4,2,1]} \else \vfill           \fillwithlines{\stretch{1}}\fi
	\item at least 80\% of the votes 
		\ifsolns \fbox{[12|8,4,2,1]} \else \vfill           \fillwithlines{\stretch{1}}\fi
	\item more than 80\% of the votes
		\ifsolns \fbox{[13|8,4,2,1]} \else \vfill           \fillwithlines{\stretch{1}}\fi
\end{enumerate}
\end{enumerate}

%</WORKSHEETS>

%<*HWHEADER>
\HOMEWORK
%</HWHEADER>

%<*HOMEWORK>

\begin{Venumerate}

  \item Alice, Bob, Charles, and Danielle are the stockholders in Alphabet Industries, Inc.
        Alice owns 252 shares, Bob owns 741 shares, Charles inherited 637 shares,
        and 412 shares are in Danielle's hands.
        As usual, each share corresponds to a vote in the stockholder's meeting.
        \begin{enumerate}
          \item If a certain type of motion requires a majority vote,
                what is the smallest number of votes needed to pass the motion?
                \solution*{%
                  \fbox{1022.}  
                  \ifgradersolns
                    (Partial credit for 1021.)
                  \fi
                }\vfill           \fillwithlines{\stretch{1}}
                
          \item A different type of motion requires a $2/3$ vote to pass.
                What is the smallest number of votes needed to pass this motion?
                \solution{\fbox{1362.}} \vfill           \fillwithlines{\stretch{1}}
          \item Using the quota you found in part (b), express the weighted voting system
                in the correct notation (with brackets and quota).
                \solution*{\fbox{$[1362 | 741, 637, 412, 252]$.}}\vfill           \fillwithlines{\stretch{1}}
        \end{enumerate}
  \item Which voters have veto power in the system $[51 | 29, 21, 8, 3, 1]$?
                \ifsolns
                  \par\soln \fbox{$P_1$ and $P_2$} (the 29 and the 21) have veto power.
                \fi             \vfill%\fillwithlines{\stretch{1}}


\hwnewpage
  \item Find all dictators, dummies, and voters with veto power in the following weighted voting systems:
        \begin{enumerate}
          \item $[51 | 20, 20, 20]$
                \solution{No dictators, no dummies, all three have veto power.}   \vfill       %\fillwithlines{\stretch{1}}
          \item $[51 | 36, 34, 23, 6]$
                \solution*{No dictators, $P_4$ (the 6) is a dummy, no one has veto power.}        \vfill     %\fillwithlines{\stretch{1}}
          \item $[25 | 27, 11, 7, 2]$
                \solution*{$P_1$ (the 27) is a dictator (and so has veto power); the rest are dummies.}   \vfill%          \fillwithlines{\stretch{1}}
          \item $[31 | 15, 13, 6, 4, 2]$
                \solution{No dictators;
                        $P_1$ and $P_2$ (the 15 and the 13) have veto power;
                        $P_5$ (the 2) is a dummy.}            \vfill% \fillwithlines{\stretch{1}}
        \end{enumerate}

%\hwnewpage
	\item In 1958, the Treaty of Rome established the European Economic Community (EEC) and instituted a
          weighted voting system for the EEC's governance.  
          The members at that time were France, Germany, Italy, Belgium, the Netherlands, and Luxembourg.  
          The three largest countries (France, Germany and Italy) were each given a vote with
          weight 4, Belgium and the Netherlands had votes of weight 2 and Luxembourg's vote had weight 1.  
          The quota was 12.
		
          What is unusual or interesting about this weighted voting system?
          \ifsolns
            \par\soln \fbox{Luxembourg is a dummy.}
          \fi
	      \vfill           \fillwithlines{\stretch{2}}

\end{Venumerate}

\ENDHOMEWORK %</HOMEWORK>

%<*WORKSHEETS>

\cleartooddpage
\section{Banzhaf Power Index} \index{power index!Banzhaf}

\begin{enumerate}

    \item A \defnstyle{coalition} is \ldots \index{coalition}
		\ifsolns
			any set of players that might join forces and vote the same way.  In principle, we can have a coalition with as few as \emph{one} player and as many as \emph{all} players.  The coalition consisting of all the players is called the \textbf{grand coalition}. Since coalitions are just sets of players, the most convenient way to describe coalitions mathematically is to use the \emph{set} notation.
		\else
			\vspace{1in}
		\fi
    \item \begin{enumerate}
            \item Consider a weighted voting system with three voters $P_1$, $P_2$, and $P_3$.
                  List all the coalitions.
                  How many are there?
                  \vfill
            \item Consider a weighted voting system with four voters $P_1$, $P_2$, $P_3$, and $P_4$.
                  List all the coalitions.
                  How many are there?
                  \vfill
            \item If a weighted voting system has $n$ voters $P_1$, $P_2$, \ldots, $P_n$,
                  how many coalitions are there?
    \ifsolns
				\fbox{$2^n-1$}
		\else
			\fillwithlines{\stretch{1}}
		\fi
          \end{enumerate}
\clearpage
    \item A \defnstyle{winning coalition} is \ldots \index{coalition!winning}
				\ifsolns
			one that has enough votes to win.  A single player coalition can be a winning coalition only when that player is a dictator, so under the assumption that there are no dictators in our weighted voting system (dictators are boring) a winning coalition must have at least two players.
		\else
			\fillwithlines{\stretch{1}}
		\fi
    \item List all the winning coalitions in the weighted voting system $[10 | 5, 4, 3, 2, 1]$.
          \vfill
    \item In the weighted voting system \[ [10 | 6, 4, 3, 2, 1],\] 
          consider \textbf{the winning coalition} $\{P_1, P_2, P_3, P_4\}$.\\
          Which voter(s) could change their minds and vote ``no'' without changing the outcome of the vote?
          Which voter(s) \emph{need} to keep voting ``yes'' in order for the motion to pass?
          \vfill
    \item A \defnstyle{critical voter} in a winning coalition is \ldots \index{critical voter}
			\ifsolns
			the coalition must have that player's votes to win.
		\else
			\fillwithlines{\stretch{1}}
		\fi

\clearpage
%\newcommand*\circled[1]{\raisebox{.5pt}{\textcircled{\raisebox{-.9pt} {#1}}}}

		
   \item Consider the voting system $[19|11,9,8,5]$.   % $[21|10,8,5,3,2]$.
         \begin{enumerate}
           \item List all the \emph{winning} coalitions.
					
					\ifsolns{
					\centering
					\tikz[inner sep=.25ex,baseline=-.75ex] \node[circle,draw] {$P_1$}; \tikz[inner sep=.25ex,baseline=-.75ex] \node[circle,draw] {$P_2$}; \ \tikz[inner sep=.25ex,baseline=-.75ex] \node[circle,draw] {$P_1$};\tikz[inner sep=.25ex,baseline=-.75ex] \node[circle,draw] {$P_3$};, \\
					\tikz[inner sep=.25ex,baseline=-.75ex] \node[circle,draw] {$P_1$};\tikz[inner sep=.25ex,baseline=-.75ex] \node {$P_2 P_3$};,\ 
					\tikz[inner sep=.25ex,baseline=-.75ex] \node[circle,draw] {$P_1$};\tikz[inner sep=.25ex,baseline=-.75ex] \node[circle, draw] {$P_2$}; \tikz[inner sep=.25ex, baseline=-.75ex] \node {$P_4$};, \ 
					\tikz[inner sep=.25ex,baseline=-.75ex] \node[circle,draw] {$P_1$};\tikz[inner sep=.25ex,baseline=-.75ex] \node[circle, draw] {$P_3$}; \tikz[inner sep=.25ex, baseline=-.75ex] \node {$P_4$};, \ 
					\tikz[inner sep=.25ex,baseline=-.75ex] \node[circle,draw] {$P_2$};\tikz[inner sep=.25ex,baseline=-.75ex] \node[circle, draw] {$P_3$}; \tikz[inner sep=.25ex, baseline=-.75ex] \node[circle, draw] {$P_4$};, \ 
				\\
					$P_1 P_2 P_3 P_4$,
					
					}\fi
                 \vfill
           \item In each winning coalition above, circle the \defnstyle{critical voters}.
           \item Count the number of times each voter is a critical voter.
                 This is called that voter's \defnstyle{Banzhaf power}. \index{power index!Banzhaf}
                 \begin{center}
                   \begin{tabular}{l|l}
                     Voter & Banzhaf power \\ \hline
										\ifsolns
										 $P_1$ & 5 \\ \hline
                     $P_2$ & 3 \\ \hline
                     $P_3$ & 3 \\ \hline
                     $P_4$ & 1 \\ \hline
                   \else
                     $P_1$ & \raisebox{0pt}[15pt][5pt]{} \\ \hline
                     $P_2$ & \raisebox{0pt}[15pt][5pt]{} \\ \hline
                     $P_3$ & \raisebox{0pt}[15pt][5pt]{} \\ \hline
                     $P_4$ & \raisebox{0pt}[15pt][5pt]{} \\ \hline
										\fi
                   \end{tabular}
                 \end{center}
           \item Add up all the voters' Banzhaf powers; this sum is called the \defnstyle{total Banzhaf power} of the voting system.
                 \vspace{0.5in}
           \item Finally, divide each voter's Banzhaf power by the total Banzhaf power.
                 The percentage that results is called the voter's \defnstyle{Banzhaf power \underline{index}}.
                 \begin{center}
                   \begin{tabular}{l|l}
                     Voter & Banzhaf power \emph{index}\\ \hline
                     $P_1$ & \raisebox{0pt}[15pt][5pt]{} \\ \hline
                     $P_2$ & \raisebox{0pt}[15pt][5pt]{} \\ \hline
                     $P_3$ & \raisebox{0pt}[15pt][5pt]{} \\ \hline
                     $P_4$ & \raisebox{0pt}[15pt][5pt]{} \\ \hline
                   \end{tabular}
                 \end{center}
         \end{enumerate}

\clearpage
	\item Calculate the Banzhaf Power Index for each voter in the weighted voting system $[51| 32, 22, 12]$.
          \vfill
    \item Make up a weighted voting system with a dummy, and calculate the Banzhaf Power Index for the dummy.
          \vfill
    \item Make up a weighted voting system with a dictator, and calculate the Banzhaf Power Index for the dictator.
          \vfill
    \item Make up a weighted voting system in which several voters have veto power.
          Calculate the Banzhaf Power Index for the voters with veto power.  What do you notice?
          \vfill
					\fillwithlines{\stretch{1}}
\clearpage
	\item The U.N. Security Council consists of 15 member countries--5 permanent members and 10 non-permanent members.  A motion can pass only if it has the vote of \emph{all five} of the permanent members plus at least four of the non-permanent members.
	\begin{enumerate}
		\item Describe the critical players in a winning coalition.  \label{UNSC1}
		\ifsolns For coalitions with 9 players, every player is critical.  For coalitions with more than 9 players, only permanent members are critical. \vfill
		\else \fillwithlines{\stretch{1}}\fi
		\item Use your answer in (a),%\ref{UNSC1}, 
		together with the fact that there are 210 nine-member winning coalitions and 638 winning coalitions with 10 or more members, to explain why the total number of times all players are critical is 5080. \label{UNSC2}
		\ifsolns \[ 210 \times 9  + 638 \times 5 = 5080\] \vfill
		\else \fillwithlines{\stretch{1}}\fi
		\item Using the results of (a) and (b), %\ref{UNSC1} and \ref{UNSC2}, 
		show that the Banzhaf power index of a permanent member is given by the ratio $848/5080$. \label{UNSC3} \ifsolns Since permanent members are critical in every winning coalition, each permanent member is critical \[210+638 = 848 \] times, \fi\vfill
		\item Using the results of (a), (b) and (c), %\ref{UNSC1}, \ref{UNSC2} and \ref{UNSC3}, 
		show that the Banzhaf power index of a non-permanent member is given by the ratio $84/5080$. 
		\ifsolns The remaining times players are critical, but are not permanent is 
		\[ 5080 - 5\times 848 = 840 \]
		This needs to be shared amoung the 10 non-permanent members so each one is critical 84 times. \fi\vfill
		\item Explain why the U.N. Security Council is equivalent to the weighed voting system in which each non-permanent member has 1 vote, each permanent member has 7 votes and the quota is 39 votes. \ifsolns 
		It just works out.  Nearly all students will not have time to complete the calculations in class. \vfill\else \fillwithlines{\stretch{1}}\fi
	\end{enumerate}
\end{enumerate}

%</WORKSHEETS>

%<*HWHEADER>
\HOMEWORK
%</HWHEADER>

%<*HOMEWORK>

\begin{Venumerate}
  \item List all the winning coalitions in the weighted voting system $[12|7,5,4,2]$.
        \solution*{
          For convenience, we list them both by $P$-number and by their weights:\par
          \[\boxed{\begin{array}{ll}
              \{P_1, P_2, P_3, P_4\} & \{7,5,4,2\} \\
              \{P_1, P_2, P_3\} & \{7,5,4\} \\
              \{P_1, P_2, P_4\} & \{7,5,2\} \\
              \{P_1, P_3, P_4\} & \{7,4,2\} \\
              \{P_1, P_2\} & \{7,5\} \\
            \end{array}}\] }
						\vfill\vfill
  \item List all the winning coalitions in the weighted voting system $[11|6,4,3,3,1]$.
        \solution{
          For convenience, we list them both by $P$-number and by their weights:\par
      \[\boxed{\begin{array}{ll}
              \{P_1, P_2, P_3, P_4, P_5\} & \{6,4,3,3,1\} \\
              \{P_1, P_2, P_3, P_4\} & \{6,4,3,3\} \\
              \{P_1, P_2, P_3, P_5\} & \{6,4,3,1\} \\
              \{P_1, P_2, P_4, P_5\} & \{6,4,3,1\} \\
              \{P_1, P_3, P_4, P_5\} & \{6,3,3,1\} \\
              \{P_2, P_3, P_4, P_5\} & \{4,3,3,1\} \\
              \{P_1, P_2, P_3\} & \{6,4,3\} \\
              \{P_1, P_2, P_4\} & \{6,4,3\} \\
              \{P_1, P_2, P_5\} & \{6,4,1\} \\
              \{P_1, P_3, P_4\} & \{6,3,3\} \\
            \end{array}}\]
        }\vfill\vfill
\vfill  \item In the weighted voting system \[ [38|22,20,17,9,5], \] consider the winning coalition
        $\{P_2, P_3, P_4, P_5\}$.
        Which voters are critical voters in this coalition?
        \solution*{$P_2$ and $P_3$ are critical voters (the 20 and the 17).}\vfill
  \vfill\item In the weighted voting system \[ [7|3,3,2,2,2,1], \] consider the winning coalition
        $\{P_1,P_3,P_4,P_6\}$.
        Which voters are critical voters in this coalition?
        \solution{$P_1$, $P_3$, and $P_4$ are critical voters (the 3, 2, and 2).}\vfill
  \vfill\item Calculate the Banzhaf Power Index for each voter in the weighted voting system $[34 | 12, 10, 7, 6]$.
        %\ifsolns
          \solution*{All four voters have equal Banzhaf power: \fbox{$0.25, 0.25, 0.25, 0.25$.}}\vfill
        %\fi
  \vfill\item Consider the voting system $[25|24,20,1]$.
        \begin{enumerate}
          \item Calculate the percentage of the total weight that each voter holds.
                \solution*{53.3\%, \quad 44.4\%, \quad 2.2\%}
          \vfill\item Calculate the Banzhaf Power Index for each voter.
                \solution*{0.60, \quad 0.20, \quad 0.20}
          \vfill\item Comparing your answers to parts (a) and (b), explain in complete sentences
                why the weight controlled by the voter is not the same thing as the power held by each voter.
                \ifsolns
                  \par\soln 
                  There are several ways to answer this question.
                  For example, voter $P_2$ has twenty times more weight than voter $P_3$,
                  but they have exactly the same power.
                  Another way to see it is that voter $P_1$ has fully three times as much power as voter $P_2$,
                  even though the weights of 24 and 20 are not that different.
                \fi
        \end{enumerate}
				
\hwnewpage
  \vfill\item Calculate the Banzhaf Power Index for each voter in the weighted voting system $[27 | 15, 7, 5]$.
        \ifsolns
          \par\soln \fbox{0.33, 0.33, 0.33}
        \fi
  \vfill\item Calculate the Banzhaf Power Index for each voter in the weighted voting system $[26 | 15, 13, 7]$.
        \solution*{0.5, \quad 0.5, \quad 0}
  \vfill\item Calculate the Banzhaf Power Index for each voter in the weighted voting system $[63|43,35,22,16]$.
        \solution{0.417,\quad 0.25\quad 0.25\quad 0.0833}
%%   \item Nassau County, New York used to be governed by a Board of Supervisors.
%%         The county had six districts, each of which one delegate to vote on county issues.
%%         The delegates' votes were weighted proportionately to the districts' population in 1964:
%%         \begin{center}
%%           \begin{tabular}{l|r}
%%             District & Weight \\ \hline
%%             Hempstead \#1 & 31 \\
%%             Hempstead \#2 & 31 \\
%%             Oyster Bay    & 28 \\
%%             North Hempstead & 21 \\
%%             Long Beach    &  2 \\
%%             Glen Cove     &  2
%%           \end{tabular}
%%         \end{center}
%%         A simple majority was needed to pass a motion.
%%         \begin{enumerate}
%%           \item Express this weighted voting system in our usual notation.
%%                 \ifsolns
%%                   \par\soln \fbox{$[ 58 | 31, 31, 28, 21, 2, 2]$}
%%                 \fi
%%           \item Calculate the Banzhaf power of each district.
%%                 \ifsolns
%%                   \par\soln A motion will pass if and only if two of the three largest voters vote for it.  Thus the Banzhaf power is: \\
%%                   \begin{tabular}[c]{l|rr}
%%                     District        & Weight & Banzhaf Power\\ \hline
%%                     Hempstead \#1   & 31     & 0.33 \\
%%                     Hempstead \#2   & 31     & 0.33 \\
%%                     Oyster Bay      & 28     & 0.33 \\
%%                     North Hempstead & 21     & 0.00 \\
%%                     Long Beach      &  2     & 0.00 \\
%%                     Glen Cove       &  2     & 0.00 \\
%%                   \end{tabular}
%%                 \fi
%%           \item What percentage of the county population lived in districts that are dummies?
%%                 \ifsolns
%%                   \par\soln Since the weights are proportionate to the population, and the total weight is 115,
%%                   we conclude that $\dfrac{21+2+2}{31+31+28+21+2+2} = \dfrac{25}{115} = \boxed{21.7\%}$
%%                   of the population had no say at all in county government.
%%                 \fi
%%           \item In 1965 John F.~Banzhaf~III argued in court that even though the weights were proportionate to population,
%%                 this system of government was unfair.
%%                 He won!
%%                 \ifsolns
%%                   \par\fbox{\emph{(No answer required.)}}
%%                 \fi
%%         \end{enumerate}
\vfill
\end{Venumerate}

\ENDHOMEWORK

\section{Shapley-Shubik Power Index} \index{power index!Shapley-Shubik}

\begin{enumerate}

    \item A \defnstyle{sequential coalition} is \ldots  \index{coalition!sequential}
			\ifsolns
				A coalition of all voters in a particular order.  The assumption is that coalitions are formed sequentially: Players join the coalition and cast their votes in an orderly sequence.
			\else
				\fillwithlines{\stretch{1}}
			\fi
    \item \begin{enumerate}
            \item Consider a weighted voting system with three voters $P_1$, $P_2$, and $P_3$.
                  List all the sequential coalitions.
                  How many are there?
									\ifsolns \par $\{P_1,P_2,P_3\}, \{P_1,P_3,P_2\},\{P_2,P_1,P_3\},\{P_3,P_2,P_1\},\{P_2,P_3,P_1\},\{P_3,P_1,P_2\}$,\fbox{6}\fi
                  \vfill
            \item Consider a weighted voting system with four voters $P_1$, $P_2$, $P_3$, and $P_4$.
                  List all the sequential coalitions.
                  How many are there? \ifsolns \fbox{24}\fi
                  \vfill
									\vfill
            \item If a weighted voting system has $n$ voters $P_1$, $P_2$, \ldots, $P_n$,
                  how many sequential coalitions are there? \ifsolns $n!$ \fi
                  
          \end{enumerate}
\clearpage
    \item A \defnstyle{pivotal player} is \ldots  \index{pivotal player}
			\ifsolns the player who contributes the votes to turn a losing coalition into a winning coalition. \fi
			\fillwithlines{\stretch{1}}
			
    \item List all the sequential coalitions in the weighted voting system $[4 | 3, 2, 1]$ and determine the pivotal player.
          \vfill
    \item In the weighted voting system $[4 | 3, 2, 1]$, 
    			Is there a voter who is always pivotal?
    			Is there a voter who is never pivotal?
          \vfill
	\ifsolns
  	 To calculate the power of a voter, count the number of times the voter is pivotal. Then divide by the total number of sequential coalitions. Calculate the power of each voter in the weighted voting system $[4 | 3, 2, 1]$.  Just like in the previous power index, the sum of all the powers must equal one.
		\fi

\clearpage
   \item Consider the voting system $[6|4,3,2,1]$.   % $[21|10,8,5,3,2]$.
         \begin{enumerate}
           \item List all the \emph{sequential} coalitions.
                 \vfill
           \item In each sequential coalition above, circle the \defnstyle{pivotal voters}.
           \item Count the number of times each voter is a pivotal voter.
                 This is called that voter's \defnstyle{Shapley-Shubik power}. \index{power index!Shapley-Shubik}
                 \begin{center}
                   \begin{tabular}{l|l}
                     Voter & Shapley-Shubik power \\ \hline
										\ifsolns
                     $P_1$ & 10 \\ \hline
                     $P_2$ & 6 \\ \hline
                     $P_3$ & 6 \\ \hline
                     $P_4$ & 2 \\ \hline
										\else
                     $P_1$ & \raisebox{0pt}[15pt][5pt]{} \\ \hline
                     $P_2$ & \raisebox{0pt}[15pt][5pt]{} \\ \hline
                     $P_3$ & \raisebox{0pt}[15pt][5pt]{} \\ \hline
                     $P_4$ & \raisebox{0pt}[15pt][5pt]{} \\ \hline
										\fi
                   \end{tabular}
                 \end{center}
           \item Add up all the voters' Shapley-Shubik powers; this sum is called the \defnstyle{total Shapley-Shubik power} of the voting system.
                 \vspace{0.5in}
           \item Finally, divide each voter's Shapley-Shubik power by the total Shapley-Shubik power.
                 The percentage that results is called the voter's \defnstyle{Shapley-Shubik power \underline{index}}.
                 \begin{center}
                   \begin{tabular}{l|l}
                     Voter & Shapley-Shubik power \emph{index}\\ \hline
                     $P_1$ & \raisebox{0pt}[15pt][5pt]{} \\ \hline
                     $P_2$ & \raisebox{0pt}[15pt][5pt]{} \\ \hline
                     $P_3$ & \raisebox{0pt}[15pt][5pt]{} \\ \hline
                     $P_4$ & \raisebox{0pt}[15pt][5pt]{} \\ \hline
                   \end{tabular}
                 \end{center}
         \end{enumerate}

\clearpage
	\item Calculate the Shapley-Shubik Power Index for each voter in the weighted voting system $[51| 32, 22, 12]$.
          \vfill
    \item Make up a weighted voting system with a dummy, and calculate the Shapley-Shubik Power Index for the dummy.
          \vfill
    \item Make up a weighted voting system with a dictator, and calculate the Shapley-Shubik Power Index for the dictator.
          \vfill
    \item Make up a weighted voting system in which several voters have veto power.
          Calculate the Shapley-Shubik Power Index for the voters with veto power.  What do you notice?
          \vfill

\clearpage
	\item In some cities the city Council operates under what is known as the, ``strong -- mayor''. Under this system the city Council can pass a motion under a simple majority, but the mayor has the power to veto the decision. The mayor's veto can then be overruled by a ``super majority''  \index{super majority} of the council members. As an example, consider the city of Ice-n-knock.  In Ice-n-knock, the city Council has four members plus a strong mayor who has a vote as well as the power to veto motion supported by a simple majority of the council members. On the other hand, the mayor cannot veto a motion supported by all four Council members. Thus, a motion can pass if the mayor +2 or more Council members supported or, alternatively, if the mayor is against it at the four council members support it. 
	
	It makes sense that under these rules, the four council members have the same amount of power, but the mayor has more. Compute the Shapley-Shubik Power Index of this weighted voting system to figure out exactly how much more.
	
	\ifsolns Focus on the last sentence. A motion can pass if the mayor +2 or more Council members supported or, alternatively, if the mayor is against it at the four council members support it. The Mayor is pivotal whenever they are in the third or fourth position in a sequential coalition.  There are 4! of each of these sequential coalitions. So -- there are a total of 5!=120 total sequential coalitions.  The Mayor is pivotal in $2\times 4! = 48$ of them. Each council member is pivotal in $\frac{120-48}{4}=18$ of them. The Mayor has power of 40\% and each Council member has a power of 15\%.
	\fi
	\vfill
	
	For purposes of comparison, calculate the Banzhaf power distribution of Ice-n-knock.
	\ifsolns This will probably not happen.  But, there are winning coalitions of 3, 4, and 5 players.  The 5 member coalition has no critical members.  There are 5 4-member coalitions.  4 of them have the Mayor as a member and they are the only critical member of those coalitions.  The fifth, without the Mayor, each member is critical. There are 6 3-member winning coalitions that contain the Mayor.  All members are critical in each of these coalitions. So, the Mayor is critical in $4+6=10$ places.  The total number of critical council members is $4 + 6*2=16$ places.  Thus, our denominator is 26, the Mayor's numerator is 10 and each Council member's numerator is 4.  The Mayor has power of 38.5\% and each Council member has a power of 15.4\%. \fi
\vfill
\end{enumerate}

%</WORKSHEETS>

%<*HWHEADER>
\HOMEWORK
%</HWHEADER>

%<*HOMEWORK>

\begin{Venumerate}
  \item List all the sequential coalitions in the weighted voting system $[16|9,8,7]$.
        %\ifsolns \par
				\solution*{
        For convenience, we list them both by $P$-number and by their weights: 
          \[\boxed{\begin{array}{ll}
              \{P_1, P_2, P_3\} & \{9,8,7\} \\
              \{P_1, P_3, P_2\} & \{9,7,8\} \\
              \{P_2, P_1, P_3\} & \{8,9,7\} \\
              \{P_2, P_3, P_1\} & \{8,7,9\} \\
              \{P_3, P_1, P_2\} & \{7,9,8\} \\
              \{P_3, P_2, P_1\} & \{7,8,9\} \\
            \end{array}}\]
						}
        %\fi
  \vfill \item List all the sequential coalitions in the weighted voting system $[51|40,30,20,10]$.
        \solution{
          For convenience, we list them both by $P$-number and by their weights:
          \[\boxed{\begin{array}{ll}
              \{P_1, P_2, P_3, P_4\} & \{40, 30, 20, 10\} \\
							\{P_1, P_2, P_4, P_3\} & \{40, 30, 10, 20\} \\
							\{P_1, P_3, P_2, P_4\} & \{40, 20, 30, 10\} \\
							\{P_1, P_3, P_4, P_2\} & \{40, 20, 10, 30\} \\
							\{P_1, P_4, P_2, P_3\} & \{40, 10, 20, 30\} \\
							\{P_1, P_4, P_3, P_2\} & \{40, 10, 30, 20\} \\
              \{P_2, P_1, P_3, P_4\} & \{30, 40, 20, 10\} \\
							\{P_2, P_1, P_4, P_3\} & \{30, 40, 10, 20\} \\
							\{P_2, P_3, P_1, P_4\} & \{30, 20, 40, 10\} \\
							\{P_2, P_3, P_4, P_1\} & \{30, 20, 10, 40\} \\
							\{P_2, P_4, P_1, P_3\} & \{30, 10, 20, 40\} \\
							\{P_2, P_4, P_3, P_1\} & \{30, 10, 40, 20\} \\
              \{P_3, P_1, P_2, P_4\} & \{20, 40, 30, 10\} \\
							\{P_3, P_1, P_4, P_2\} & \{20, 40, 10, 30\} \\
							\{P_3, P_2, P_1, P_4\} & \{20, 30, 40, 10\} \\
							\{P_3, P_2, P_4, P_1\} & \{20, 30, 10, 40\} \\
							\{P_3, P_4, P_1, P_2\} & \{20, 10, 40, 30\} \\
							\{P_3, P_4, P_2, P_1\} & \{20, 10, 30, 40\} \\
							\{P_4, P_1, P_2, P_3\} & \{10, 40, 30, 20\} \\
							\{P_4, P_1, P_3, P_2\} & \{10, 40, 20, 30\} \\
							\{P_4, P_2, P_1, P_3\} & \{10, 30, 40, 20\} \\
							\{P_4, P_2, P_3, P_1\} & \{10, 30, 20, 40\} \\
							\{P_4, P_3, P_1, P_2\} & \{10, 20, 40, 30\} \\
							\{P_4, P_3, P_2, P_1\} & \{10, 20, 30, 40\} \\
            \end{array}}\]
        } \vfill
%  \item In the weighted voting system $[38|22,20,17,9,5]$, consider the winning coalition
%        $\{P_2, P_3, P_4, P_5\}$.
%        Which voters are critical voters in this coalition?
%        \solution*{$P_2$ and $P_3$ are critical voters (the 20 and the 17).}
%  \item In the weighted voting system $[7|3,3,2,2,2,1]$, consider the winning coalition
%        $\{P_1,P_3,P_4,P_6\}$.
%        Which voters are critical voters in this coalition?
%        \solution{$P_1$, $P_3$, and $P_4$ are critical voters (the 3, 2, and 2).}
%  \item Calculate the Shapley-Shubik Power Index for each voter in the weighted voting system $[34 | 12, 10, 7, 6]$.
%        \ifsolns
%          \solution*{All four voters have equal Shapley-Shubik power: \fbox{$0.25, 0.25, 0.25, 0.25$.}}
%        \fi
  \item Consider the voting system $[25|24,20,1]$.
        \begin{enumerate}
          \item Calculate the percentage of the total weight that each voter holds.
                \solution*{53.3\%, \quad 44.4\%, \quad 2.2\%}
          \vfill \item Calculate the Shapley-Shubik Power Index for each voter.
                \solution*{0.667, \quad 0.167, \quad 0.167}
          \vfill \item Comparing your answers to parts (a) and (b), explain in complete sentences
                why the weight controlled by the voter is not the same thing as the power held by each voter.
                \ifsolns
                  \par\soln 
                  There are several ways to answer this question.
                  For example, voter $P_2$ has twenty times more weight than voter $P_3$,
                  but they have exactly the same power.
                  Another way to see it is that voter $P_1$ has fully three times as much power as voter $P_2$,
                  even though the weights of 24 and 20 are not that different.
                \fi
        \end{enumerate}
  \vfill 
	\hwnewpage
	\item Calculate the Shapley-Shubik Power Index for each voter in the weighted voting system $[15|16,8,4,1]$.
        \ifsolns
          \par\soln \fbox{1,0,0}
        \fi
  \vfill \item Calculate the Shapley-Shubik Power Index for each voter in the weighted voting system $[24|16,8,4,1]$.
        \solution*{0.5,  0.5,  0,  0}
  \vfill \item Calculate the Shapley-Shubik Power Index for each voter in the weighted voting system $[28|16,8,4,1]$.
			\solution*{0.33, 0.33, 0.33, 0}
%        \solution{0.417,\quad 0.25\quad 0.25\quad 0.0833}
%%   \item Nassau County, New York used to be governed by a Board of Supervisors.
%%         The county had six districts, each of which one delegate to vote on county issues.
%%         The delegates' votes were weighted proportionately to the districts' population in 1964:
%%         \begin{center}
%%           \begin{tabular}{l|r}
%%             District & Weight \\ \hline
%%             Hempstead \#1 & 31 \\
%%             Hempstead \#2 & 31 \\
%%             Oyster Bay    & 28 \\
%%             North Hempstead & 21 \\
%%             Long Beach    &  2 \\
%%             Glen Cove     &  2
%%           \end{tabular}
%%         \end{center}
%%         A simple majority was needed to pass a motion.
%%         \begin{enumerate}
%%           \item Express this weighted voting system in our usual notation.
%%                 \ifsolns
%%                   \par\soln \fbox{$[ 58 | 31, 31, 28, 21, 2, 2]$}
%%                 \fi
%%           \item Calculate the Banzhaf power of each district.
%%                 \ifsolns
%%                   \par\soln A motion will pass if and only if two of the three largest voters vote for it.  Thus the Banzhaf power is: \\
%%                   \begin{tabular}[c]{l|rr}
%%                     District        & Weight & Banzhaf Power\\ \hline
%%                     Hempstead \#1   & 31     & 0.33 \\
%%                     Hempstead \#2   & 31     & 0.33 \\
%%                     Oyster Bay      & 28     & 0.33 \\
%%                     North Hempstead & 21     & 0.00 \\
%%                     Long Beach      &  2     & 0.00 \\
%%                     Glen Cove       &  2     & 0.00 \\
%%                   \end{tabular}
%%                 \fi
%%           \item What percentage of the county population lived in districts that are dummies?
%%                 \ifsolns
%%                   \par\soln Since the weights are proportionate to the population, and the total weight is 115,
%%                   we conclude that $\dfrac{21+2+2}{31+31+28+21+2+2} = \dfrac{25}{115} = \boxed{21.7\%}$
%%                   of the population had no say at all in county government.
%%                 \fi
%%           \item In 1965 John F.~Banzhaf~III argued in court that even though the weights were proportionate to population,
%%                 this system of government was unfair.
%%                 He won!
%%                 \ifsolns
%%                   \par\fbox{\emph{(No answer required.)}}
%%                 \fi
%%         \end{enumerate}
\vfill
\end{Venumerate}


\ENDHOMEWORK %</HOMEWORK>

%<*WORKSHEETS>

\cleartooddpage
\section{Study Guide}


To prepare for the exam over this chapter,
you should review the in-class worksheets and homework.
Be ready to do the kind of problems you faced on the homework.

As a general guide, I recommend reviewing the following topics.

\begin{enumerate}
  \item Know how to read a table of voters' preferences
  \item Calculate the winner according to
        \begin{enumerate}
          \item Plurality
          \item Borda count
          \item Plurality with elimination
          \item Pairwise comparison
        \end{enumerate}
  \item Devise preference tables that satisfy given conditions
        (e.g.,~``Come up with a preference table where the pairwise comparison test produces no winner.'')
  \item Weighted voting
        \begin{enumerate}
          \item Know how weighted voting on yes/no motions works.
          \item Understand the notation $[q|w_1,\ldots,w_n]$.
          \item Given a weighted voting system, find any dictators, dummies, or voters with veto power.
        \end{enumerate}
  \item Construct weighted voting systems that satisfy given conditions
        (e.g.,~``Come up with a weighted voting system where two people have veto power.'')
  \item Calculate the Banzhaf Power Index for the voters in a weighted voting system.
  \item Calculate the Schaply-Shubik Power Index for the voters in a weighted voting system.
\end{enumerate}

%</WORKSHEETS>

\endinput
