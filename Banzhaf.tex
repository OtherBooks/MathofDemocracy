\section{Banzhaf Power Index} \index{power index!Banzhaf} \label{sec:Banzhaf}

\begin{enumerate}

    \item A \defnstyle{coalition} is \ldots \index{coalition}
		\ifsolns
			any set of players that might join forces and vote the same way.  In principle, we can have a coalition with as few as \emph{one} player and as many as \emph{all} players.  The coalition consisting of all the players is called the \textbf{grand coalition}. Since coalitions are just sets of players, the most convenient way to describe coalitions mathematically is to use the \emph{set} notation.
		\else
			\vspace{1in}
		\fi
    \item \begin{enumerate}
            \item Consider a weighted voting system with three voters $P_1$, $P_2$, and $P_3$.
                  List all the coalitions.
                  How many are there?
                  \vfill
            \item Consider a weighted voting system with four voters $P_1$, $P_2$, $P_3$, and $P_4$.
                  List all the coalitions.
                  How many are there?
                  \vfill
            \item If a weighted voting system has $n$ voters $P_1$, $P_2$, \ldots, $P_n$,
                  how many coalitions are there?
    \ifsolns
				\fbox{$2^n-1$}
		\else
			\fillwithlines{\stretch{1}}
		\fi
          \end{enumerate}
\clearpage
    \item A \defnstyle{winning coalition} is \ldots \index{coalition!winning}
				\ifsolns
			one that has enough votes to win.  A single player coalition can be a winning coalition only when that player is a dictator, so under the assumption that there are no dictators in our weighted voting system (dictators are boring) a winning coalition must have at least two players.
		\else
			\fillwithlines{\stretch{1}}
		\fi
    \item List all the winning coalitions in the weighted voting system $[10 | 5, 4, 3, 2, 1]$.
          \vfill
    \item In the weighted voting system \[ [10 | 6, 4, 3, 2, 1],\] 
          consider \textbf{the winning coalition} $\{P_1, P_2, P_3, P_4\}$.\\
          Which voter(s) could change their minds and vote ``no'' without changing the outcome of the vote?
          Which voter(s) \emph{need} to keep voting ``yes'' in order for the motion to pass?
          \vfill
    \item A \defnstyle{critical voter} in a winning coalition is \ldots \index{critical voter}
			\ifsolns
			the coalition must have that player's votes to win.
		\else
			\fillwithlines{\stretch{1}}
		\fi

\clearpage
%\newcommand*\circled[1]{\raisebox{.5pt}{\textcircled{\raisebox{-.9pt} {#1}}}}

		
   \item Consider the voting system $[19|11,9,8,5]$.   % $[21|10,8,5,3,2]$.
         \begin{enumerate}
           \item List all the \emph{winning} coalitions.
					
					\ifsolns{
					\centering
					\tikz[inner sep=.25ex,baseline=-.75ex] \node[circle,draw] {$P_1$}; \tikz[inner sep=.25ex,baseline=-.75ex] \node[circle,draw] {$P_2$}; \ \tikz[inner sep=.25ex,baseline=-.75ex] \node[circle,draw] {$P_1$};\tikz[inner sep=.25ex,baseline=-.75ex] \node[circle,draw] {$P_3$};, \\
					\tikz[inner sep=.25ex,baseline=-.75ex] \node[circle,draw] {$P_1$};\tikz[inner sep=.25ex,baseline=-.75ex] \node {$P_2 P_3$};,\ 
					\tikz[inner sep=.25ex,baseline=-.75ex] \node[circle,draw] {$P_1$};\tikz[inner sep=.25ex,baseline=-.75ex] \node[circle, draw] {$P_2$}; \tikz[inner sep=.25ex, baseline=-.75ex] \node {$P_4$};, \ 
					\tikz[inner sep=.25ex,baseline=-.75ex] \node[circle,draw] {$P_1$};\tikz[inner sep=.25ex,baseline=-.75ex] \node[circle, draw] {$P_3$}; \tikz[inner sep=.25ex, baseline=-.75ex] \node {$P_4$};, \ 
					\tikz[inner sep=.25ex,baseline=-.75ex] \node[circle,draw] {$P_2$};\tikz[inner sep=.25ex,baseline=-.75ex] \node[circle, draw] {$P_3$}; \tikz[inner sep=.25ex, baseline=-.75ex] \node[circle, draw] {$P_4$};, \ 
				\\
					$P_1 P_2 P_3 P_4$,
					
					}\fi
                 \vfill
           \item In each winning coalition above, circle the \defnstyle{critical voters}.
           \item Count the number of times each voter is a critical voter.
                 This is called that voter's \defnstyle{Banzhaf power}. \index{power index!Banzhaf}
                 \begin{center}
                   \begin{tabular}{l|l}
                     Voter & Banzhaf power \\ \hline
										\ifsolns
										 $P_1$ & 5 \\ \hline
                     $P_2$ & 3 \\ \hline
                     $P_3$ & 3 \\ \hline
                     $P_4$ & 1 \\ \hline
                   \else
                     $P_1$ & \raisebox{0pt}[15pt][5pt]{} \\ \hline
                     $P_2$ & \raisebox{0pt}[15pt][5pt]{} \\ \hline
                     $P_3$ & \raisebox{0pt}[15pt][5pt]{} \\ \hline
                     $P_4$ & \raisebox{0pt}[15pt][5pt]{} \\ \hline
										\fi
                   \end{tabular}
                 \end{center}
           \item Add up all the voters' Banzhaf powers; this sum is called the \defnstyle{total Banzhaf power} of the voting system.
                 \vspace{0.5in}
           \item Finally, divide each voter's Banzhaf power by the total Banzhaf power.
                 The percentage that results is called the voter's \defnstyle{Banzhaf power \underline{index}}.
                 \begin{center}
                   \begin{tabular}{l|l}
                     Voter & Banzhaf power \emph{index}\\ \hline
                     $P_1$ & \raisebox{0pt}[15pt][5pt]{} \\ \hline
                     $P_2$ & \raisebox{0pt}[15pt][5pt]{} \\ \hline
                     $P_3$ & \raisebox{0pt}[15pt][5pt]{} \\ \hline
                     $P_4$ & \raisebox{0pt}[15pt][5pt]{} \\ \hline
                   \end{tabular}
                 \end{center}
         \end{enumerate}

\clearpage
	\item Calculate the Banzhaf Power Index for each voter in the weighted voting system $[51| 32, 22, 12]$.
          \vfill
    \item Make up a weighted voting system with a dummy, and calculate the Banzhaf Power Index for the dummy.
          \vfill
    \item Make up a weighted voting system with a dictator, and calculate the Banzhaf Power Index for the dictator.
          \vfill
    \item Make up a weighted voting system in which several voters have veto power.
          Calculate the Banzhaf Power Index for the voters with veto power.  What do you notice?
          \vfill
					\fillwithlines{\stretch{1}}
\clearpage
	\item The U.N. Security Council consists of 15 member countries--5 permanent members and 10 non-permanent members.  A motion can pass only if it has the vote of \emph{all five} of the permanent members plus at least four of the non-permanent members.
	\begin{enumerate}
		\item Describe the critical players in a winning coalition.  \label{UNSC1}
		\ifsolns For coalitions with 9 players, every player is critical.  For coalitions with more than 9 players, only permanent members are critical. \vfill
		\else \fillwithlines{\stretch{1}}\fi
		\item Use your answer in (a),%\ref{UNSC1}, 
		together with the fact that there are 210 nine-member winning coalitions and 638 winning coalitions with 10 or more members, to explain why the total number of times all players are critical is 5080. \label{UNSC2}
		\ifsolns \[ 210 \times 9  + 638 \times 5 = 5080\] \vfill
		\else \fillwithlines{\stretch{1}}\fi
		\item Using the results of (a) and (b), %\ref{UNSC1} and \ref{UNSC2}, 
		show that the Banzhaf power index of a permanent member is given by the ratio $848/5080$. \label{UNSC3} \ifsolns Since permanent members are critical in every winning coalition, each permanent member is critical \[210+638 = 848 \] times, \fi\vfill
		\item Using the results of (a), (b) and (c), %\ref{UNSC1}, \ref{UNSC2} and \ref{UNSC3}, 
		show that the Banzhaf power index of a non-permanent member is given by the ratio $84/5080$. 
		\ifsolns The remaining times players are critical, but are not permanent is 
		\[ 5080 - 5\times 848 = 840 \]
		This needs to be shared amoung the 10 non-permanent members so each one is critical 84 times. \fi\vfill
		\item Explain why the U.N. Security Council is equivalent to the weighed voting system in which each non-permanent member has 1 vote, each permanent member has 7 votes and the quota is 39 votes. \ifsolns 
		It just works out.  Nearly all students will not have time to complete the calculations in class. \vfill\else \fillwithlines{\stretch{1}}\fi
	\end{enumerate}
\end{enumerate}

%</WORKSHEETS>

%<*HWHEADER>
\HOMEWORK
%</HWHEADER>

%<*HOMEWORK>

\begin{Venumerate}
  \item List all the winning coalitions in the weighted voting system $[12|7,5,4,2]$.
        \solution*{
          For convenience, we list them both by $P$-number and by their weights:\par
          \[\boxed{\begin{array}{ll}
              \{P_1, P_2, P_3, P_4\} & \{7,5,4,2\} \\
              \{P_1, P_2, P_3\} & \{7,5,4\} \\
              \{P_1, P_2, P_4\} & \{7,5,2\} \\
              \{P_1, P_3, P_4\} & \{7,4,2\} \\
              \{P_1, P_2\} & \{7,5\} \\
            \end{array}}\] }
						\vfill\vfill
  \item List all the winning coalitions in the weighted voting system $[11|6,4,3,3,1]$.
        \solution{
          For convenience, we list them both by $P$-number and by their weights:\par
      \[\boxed{\begin{array}{ll}
              \{P_1, P_2, P_3, P_4, P_5\} & \{6,4,3,3,1\} \\
              \{P_1, P_2, P_3, P_4\} & \{6,4,3,3\} \\
              \{P_1, P_2, P_3, P_5\} & \{6,4,3,1\} \\
              \{P_1, P_2, P_4, P_5\} & \{6,4,3,1\} \\
              \{P_1, P_3, P_4, P_5\} & \{6,3,3,1\} \\
              \{P_2, P_3, P_4, P_5\} & \{4,3,3,1\} \\
              \{P_1, P_2, P_3\} & \{6,4,3\} \\
              \{P_1, P_2, P_4\} & \{6,4,3\} \\
              \{P_1, P_2, P_5\} & \{6,4,1\} \\
              \{P_1, P_3, P_4\} & \{6,3,3\} \\
            \end{array}}\]
        }\vfill\vfill
\vfill  \item In the weighted voting system \[ [38|22,20,17,9,5], \] consider the winning coalition
        $\{P_2, P_3, P_4, P_5\}$.
        Which voters are critical voters in this coalition?
        \solution*{$P_2$ and $P_3$ are critical voters (the 20 and the 17).}\vfill
  \vfill\item In the weighted voting system \[ [7|3,3,2,2,2,1], \] consider the winning coalition
        $\{P_1,P_3,P_4,P_6\}$.
        Which voters are critical voters in this coalition?
        \solution{$P_1$, $P_3$, and $P_4$ are critical voters (the 3, 2, and 2).}\vfill
  \vfill\item Calculate the Banzhaf Power Index for each voter in the weighted voting system $[34 | 12, 10, 7, 6]$.
        %\ifsolns
          \solution*{All four voters have equal Banzhaf power: \fbox{$0.25, 0.25, 0.25, 0.25$.}}\vfill
        %\fi
  \vfill\item Consider the voting system $[25|24,20,1]$.
        \begin{enumerate}
          \item Calculate the percentage of the total weight that each voter holds.
                \solution*{53.3\%, \quad 44.4\%, \quad 2.2\%}
          \vfill\item Calculate the Banzhaf Power Index for each voter.
                \solution*{0.60, \quad 0.20, \quad 0.20}
          \vfill\item Comparing your answers to parts (a) and (b), explain in complete sentences
                why the weight controlled by the voter is not the same thing as the power held by each voter.
                \ifsolns
                  \par\soln 
                  There are several ways to answer this question.
                  For example, voter $P_2$ has twenty times more weight than voter $P_3$,
                  but they have exactly the same power.
                  Another way to see it is that voter $P_1$ has fully three times as much power as voter $P_2$,
                  even though the weights of 24 and 20 are not that different.
                \fi
        \end{enumerate}
				
\hwnewpage
  \vfill\item Calculate the Banzhaf Power Index for each voter in the weighted voting system $[27 | 15, 7, 5]$.
        \ifsolns
          \par\soln \fbox{0.33, 0.33, 0.33}
        \fi
  \vfill\item Calculate the Banzhaf Power Index for each voter in the weighted voting system $[26 | 15, 13, 7]$.
        \solution*{0.5, \quad 0.5, \quad 0}
  \vfill\item Calculate the Banzhaf Power Index for each voter in the weighted voting system $[63|43,35,22,16]$.
        \solution{0.417,\quad 0.25\quad 0.25\quad 0.0833}
%%   \item Nassau County, New York used to be governed by a Board of Supervisors.
%%         The county had six districts, each of which one delegate to vote on county issues.
%%         The delegates' votes were weighted proportionately to the districts' population in 1964:
%%         \begin{center}
%%           \begin{tabular}{l|r}
%%             District & Weight \\ \hline
%%             Hempstead \#1 & 31 \\
%%             Hempstead \#2 & 31 \\
%%             Oyster Bay    & 28 \\
%%             North Hempstead & 21 \\
%%             Long Beach    &  2 \\
%%             Glen Cove     &  2
%%           \end{tabular}
%%         \end{center}
%%         A simple majority was needed to pass a motion.
%%         \begin{enumerate}
%%           \item Express this weighted voting system in our usual notation.
%%                 \ifsolns
%%                   \par\soln \fbox{$[ 58 | 31, 31, 28, 21, 2, 2]$}
%%                 \fi
%%           \item Calculate the Banzhaf power of each district.
%%                 \ifsolns
%%                   \par\soln A motion will pass if and only if two of the three largest voters vote for it.  Thus the Banzhaf power is: \\
%%                   \begin{tabular}[c]{l|rr}
%%                     District        & Weight & Banzhaf Power\\ \hline
%%                     Hempstead \#1   & 31     & 0.33 \\
%%                     Hempstead \#2   & 31     & 0.33 \\
%%                     Oyster Bay      & 28     & 0.33 \\
%%                     North Hempstead & 21     & 0.00 \\
%%                     Long Beach      &  2     & 0.00 \\
%%                     Glen Cove       &  2     & 0.00 \\
%%                   \end{tabular}
%%                 \fi
%%           \item What percentage of the county population lived in districts that are dummies?
%%                 \ifsolns
%%                   \par\soln Since the weights are proportionate to the population, and the total weight is 115,
%%                   we conclude that $\dfrac{21+2+2}{31+31+28+21+2+2} = \dfrac{25}{115} = \boxed{21.7\%}$
%%                   of the population had no say at all in county government.
%%                 \fi
%%           \item In 1965 John F.~Banzhaf~III argued in court that even though the weights were proportionate to population,
%%                 this system of government was unfair.
%%                 He won!
%%                 \ifsolns
%%                   \par\fbox{\emph{(No answer required.)}}
%%                 \fi
%%         \end{enumerate}
\vfill
\end{Venumerate}

\ENDHOMEWORK
